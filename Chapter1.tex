\chapter[Impact forces of water drops falling on superhydrophobic surfaces]{Impact forces of water drops falling on superhydrophobic surfaces\footnote{Submitted as: Bin Zhang, \textbf{Vatsal Sanjay}, Songlin Shi, Yinggang Zhao, Cunjing Lv, Xi-Qiao Feng, and Detlef Lohse, \textit{Impact forces of water drops falling on superhydrophobic surfaces}, Phys. Rev. Let. (2022). Experiments are done by Zhang and Lv; simulations by Sanjay; analysis and writing by Zhang, Sanjay, Lv, and Lohse; and supervision by Lv and Lohse. Proofread by everyone.}}
\label{chap:DropForces}

\chaptermark{Drop impact forces} %this is used at the top of your page to denote the chapter, used in cases where the title is long

A falling liquid drop, after impact on a rigid substrate,  deforms and spreads, owing to the normal reaction force. Subsequently, if the substrate is non-wetting, the drop retracts and then jumps off. As we show here, not only is the impact itself associated with a distinct peak in the temporal evolution of the normal force, but also the jump-off, which was hitherto unknown. We characterize both peaks and elucidate how they relate to the different stages of the drop impact process. The time at which the second peak appears coincides with the formation of a Worthington jet, emerging through flow-focusing.  Even low-velocity impacts can lead to a surprisingly high second peak in the normal force, even larger than the first one, namely when the Worthington jet becomes singular due to the collapse of an air cavity in the drop. 

\clearpage
% ------------------------------------------------------------------

\section{Introduction}

In 1876-77, Arthur Mason Worthington \cite{worthington1877xxviii, worthington1877second} published the first photographs of the drop impact process, stimulating artists and researchers alike for an almost one-and-a-half century. Such drop impacts  on solid surfaces are highly relevant from an application point of view, namely in inkjet printing \cite{lohse2022fundamental}, spray coating \cite{kim2007spray}, criminal forensics \cite{smith2018influence}, and many other industrial and natural processes \cite{josserand2016drop, yarin2006drop, yarin2017}. For most of these applications, the drop impact forces, which are the subject of this chapter, can lead to serious unwanted consequences, such as soil erosion \cite{nearing1986} or the damage of engineered surfaces \cite{ahmad2013, amirzadeh2017, gohardani2011}. A thorough understanding of the drop impact forces is thus needed to develop countermeasures against these damages \cite{cheng2021drop}. Consequently, recent studies have analyzed the temporal evolution of these forces \cite{li2014, soto2014, philippi2016, zhang2017, Gordillo2018, mitchell2019, zhang2019}. 

These studies were, however, up to now limited to wetting scenarios. Then, not surprisingly, the moment of the drop touch-down \cite{wagner1932stoss, philippi2016} manifests itself in a pronounced peak in the temporal evolution of the drop impact force, whereas this force is much smaller during droplet spreading \cite{yarin2006drop, wildeman2016spreading}. For the non-wetting case, i.e., for superhydrophobic surfaces, the drop impact dynamics is much richer: then, after reaching its maximal diameter, the drop recoils \cite{bergeron2001water} and can generate an upward, so-called  Worthington jet \cite{worthington1877xxviii, bartolo2006singular}. Ultimately, the drop can even ricochet off the superhydrophobic surface \cite{richard2002contact}. Such spectacular water repellency can occur  in nature \cite{onda1996super, lafuma2003, callies2005water}, and has various technological applications \cite{tuteja2007, cho2016, liu2017, hao2016, wu2020}, including on moving substrates \cite{zhan2021}, where the droplet dynamics is even richer. This feature of superhydrophobicity however is volatile and can fail due to external disturbance such as pressure \cite{lafuma2003, callies2005water, sbragaglia2007, li2017}, evaporation \cite{tsai2010, chen2012, papadopoulos2013},  mechanical vibration \cite{bormashenko2007}, or the impact forces of prior droplets \cite{bartolo2006bouncing}.   

This chapter extends the studies on drop impact forces to the impact on superhydrophobic surfaces. Our key result is that then, next to the first above-mentioned peak in the drop impact force at drop touch-down, a {\it second peak}  in the drop impact force occurs, which under certain conditions can be even more pronounced than the first peak. The physical origin of the second peak lies in momentum conservation: when at the final phase of droplet recoil, the above-mentioned upward Worthington jet forms, momentum conservation also leads to a downward jet inside the drop \cite{lohse2004impact, lohse2018, lee2020downward, mitra2021bouncing}. It manifests itself in the second peak in the temporal evolution of the force on the substrate. Using both experiments and direct numerical simulations (DNS) with the volume-of-fluid method  \cite{basiliskpopinet1}, we will elucidate the physics of this very rich dynamical process and study its dependence on the control parameters.  

This chapter is organized as follow: \S~\ref{Ch2:SecSetup} briefly describes the experimental and numerical setups followed by \S~\ref{Ch2:SecForceTime} that correlates the temporal variation of the normal reaction with different stages of the drop impact process. We then elucidate the dependence of Weber number on the characteristic times (\S~\ref{Ch2:SecTimes}) and the two peaks in the transient normal reaction force (\S~\ref{Ch2:SecFirstPeak} and~\ref{Ch2:SecSecondPeak}). The chapter culminates in conclusions and an outlook in \S~\ref{Ch2:SecConclusion}.

\begin{figure}
	\centering
	\includegraphics[width=\textwidth]{Ch1Figures/Figure1_schematic_v3.eps}
	\caption{(a) Experimental setup: a water drop of diameter $D$ impacts a superhydrophobic quartz plate at velocity $V$. (b) Axi-symmetric simulation domain with the appropriate boundary conditions and relevant material and flow properties. The boundaries are kept far away from the drop to avoid feedback ($\mathcal{L}_\text{max} \gg D$).}
	\label{Ch2:Fig1Schematic}
\end{figure}

\begin{sidewaysfigure}
	\centering
	\includegraphics[width=175mm]{Ch1Figures/Figure1_dynamics_v3.eps}
	\caption{(a) Numerical results for a drop impact dynamics for a $D = 2.05\,\si{\milli\meter}$ diameter water drop falling at a speed of $V = 1.20\,\si{\meter}/\si{\second}$: $t =$ (i) $0\,\si{\milli\second}$ (touch-down), (ii) $0.37\,\si{\milli\second}$, (iii) $2.5\,\si{\milli\second}$, (iv) $3.93\,\si{\milli\second}$, (v) $4.63\,\si{\milli\second}$, and (vi) $5.25\,\si{\milli\second}$. The left part of each numerical snapshot shows the dimensionless viscous dissipation function, $\tilde{\xi}$ on a $\log_{10}$ scale and the right part the velocity field magnitude normalized with the impact velocity. The black velocity vectors are plotted in the center of mass reference frame of the drop to elucidate the internal flow. (b)  Spreading diameter $D(t)$ and impact force $F(t)$ on the substrate as function of time: comparison between experiments  and simulations ($\Wen = 40$). The insets show representative snapshots at specific time instants overlaid with the drop boundaries from simulations in orange, revealing good agreement. $F_1 \approx 5.1\,\si{\milli\newton}$ and  $F_2 \approx 2.3\,\si{\milli\newton}$ are the two peaks of the normal  force $F(t)$ at $t_1 \approx 0.37\,\si{\milli\second}$ and $t_2 \approx 4.63\,\si{\milli\second}$, respectively. $t_m$ is the moment corresponding to the maximum spreading of the drop and $t_3$ represents the end of contact ($F = 0$). $D_m$ and $D_2$ are the spreading diameters of the drop at $t_m$ and $t_2$, respectively. Also see supplemental movie~{\color{Myfig} 1}.}
	\label{Ch2:Fig1Dynamics}
\end{sidewaysfigure}

\section{Setup}\label{Ch2:SecSetup}

The experimental setup is sketched in figure~\ref{Ch2:Fig1Schematic}(a). A water drop impacts a quartz plate whose upper surface is coated with silanized silica nanobeads with diameter of $20\,\si{\nano\meter}$ (Glaco Mirror Coat Zero; Soft99) \cite{li2017, gauthier2015}  to attain superhydrophobicity. We directly measure the  impact force $F(t)$ by synchronizing high-speed photography with fast force sensing (also see \S~\ref{sec:ExpMethods} for details of the experimental setup). 

In DNS (figure~\ref{Ch2:Fig1Schematic}b), ideal superhydrophobicity is maintained by assuming that a thin air layer is present between the drop and the substrate \cite[][also see chapters~\ref{chap:DropViscousBouncing}, and~\ref{chap:DropOnDrop}]{ramirez2020lifting}, and forces are calculated by integrating the pressure field ($p$) at the substrate,

\begin{equation}
	\label{Ch2:Eqn::force2}
	\boldsymbol{F}(t) = F(t) \boldsymbol{\hat{z}} = \left(\int_\mathcal{A} \left(p-p_0\right)\mathrm{d}\mathcal{A}\right)\boldsymbol{\hat{z}},
\end{equation}

\noindent where $p_0$ is the ambient pressure. Furthermore, $\mathcal{A}$ and $\boldsymbol{\hat{z}}$ are the area of the superhydrophobic substrate and a unit vector perpendicular to it, respectively (also see appendix~\ref{sec:NumMethods} for details of the simulation setups). 

The initial drop diameter $D$ ($2.05\,\si{\milli\meter} \le D \le 2.76\,\si{\milli\meter}$)\footnote{Note that we use diameter of the drop as the length scale in this chapter contrary to its radius that we use throughout this thesis. We do so for ease of comparison with the earlier work of \citet{Gordillo2018}, see figures~\ref{Ch2:Fig2Times} and~\ref{Ch2:FigMain}.} and the impact velocity $V$ ($0.38\,\si{\meter}/\si{\second} \le V \le 2.96\,\si{\meter}/\si{\second}$) are independently controlled. The drop material properties are kept constant (density $\rho_d = 998\,\si{\kilogram}/\si{\meter}^{3}$, surface tension coefficient $\gamma = 73\,\si{\milli\newton}/\si{\meter}$, and dynamic viscosity $\eta_d = 1.0\,\si{\milli\pascal}\si{\second}$). All experiments were carried out at ambient air pressure and temperature. The Weber number (ratio of drop inertia to capillary pressure) $\Wen \equiv \rho_d V^2 D / \gamma$ ranges between $1 - 400$ and the Reynolds number (ratio of inertial to viscous stresses) $\Ren \equiv \rho_d V D / \eta_d \approx 800\,\,\text{to}\,\,10^5$.  Note that for our simulations, we keep the drop Ohnesorge number (ratio of inertio-capillary to inertio-viscous timescales) $\Ohn \equiv \eta_d/\left(\rho_d\gamma D\right)^{1/2}$ constant at $0.0025$ to mimic $2\,\si{\milli\meter}$ diameter water drops. 

\section{Formation of a second peak in the force}\label{Ch2:SecForceTime}

In this section, we elucidate the temporal variation of the normal reaction force and the corresponding drop impact dynamics. Figure~\ref{Ch2:Fig1Dynamics}(a) illustrates the different stages of the drop impact process for $\Wen = 40$, and figure~\ref{Ch2:Fig1Dynamics}(b) quantifies the spreading diameter $D(t)$ (maximum width of the drop at time $t$) and the normal force $F(t)$ (also see supplemental movie~{\color{Myfig} 1}). Note the remarkable quantitative agreement between the experimental and the numerical data for both $D(t)$ and $F(t)$, giving credibility to both. As the drop touches the surface (figure~\ref{Ch2:Fig1Dynamics}a-i), the normal force $F(t)$ increases sharply to reach the first peak with amplitude $F_1 \approx 5.1\,\si{\milli\newton}$ in a very short time $t_1 \approx 0.37\,\si{\milli\second}$ (figure~\ref{Ch2:Fig1Dynamics}a-ii). At this instant, the spreading diameter $D(t)$ is equal to the initial drop diameter $D$, $D (t_1) \approx D$ \cite{philippi2016, zhang2017, Gordillo2018, mitchell2019, zhang2019}. Subsequently, the normal force reduces at a relatively slow rate to a minimum ($\approx 0\,\si{\milli\newton}$) at $t_m \approx 2.5\,\si{\milli\second}$. Meanwhile the drop reaches a maximum spreading diameter $D(t_m) = D_{m}$  (figure~\ref{Ch2:Fig1Dynamics}a-iii). The force profile $F(t)$, until this instant, is very close to that on a hydrophilic surface (see \S~\ref{Ch2:SecPhobicPhilic}). However, contrary to the wetting scenario, on superhydrophobic substrates, the drop starts to retract, creating high local viscous dissipation in the neck region connecting the drop with its rim (figure~\ref{Ch2:Fig1Dynamics}a-iii,iv). Through this phase of retraction, the normal reaction force is small, but shows several oscillations owing to traveling capillary waves for $2.5\,\si{\milli\second} < t < 3.8\,\si{\milli\second}$ (figure~\ref{Ch2:Fig1Dynamics}b).  The drop retraction and the traveling capillary waves lead to flow focusing at the axis of symmetry, creating the Worthington jet (figure~\ref{Ch2:Fig1Dynamics}a-iv,v) and hence also the opposite momentum jet that results in an increase in the normal force $F(t)$. Consequently, the hitherto unknown second peak appears, here with an amplitude $F_2 \approx 2.3\,\si{\milli\newton}$ and at time $t_2 \approx 4.63\,\si{\milli\second}$. Lastly, the normal force $F(t)$ decays slowly (figure~\ref{Ch2:Fig1Dynamics}a-v,vi) to zero, finally vanishing at $t_3 \approx 8.84\,\si{\milli\second}$. This time instant $t_3$ is a much better estimate for the drop contact time as compared to the one observed at complete detachment from side view images which is about $2\,\si{\milli\second}$ longer in this case \cite{richard2002contact, chantelot_lohse_2021}. Therefore, in summary, here we have identified the mechanism for the formation of the second peak in the normal force and four different characteristic times,  $t_1$, $t_m$, $t_2$, and $t_3$ (figure~\ref{Ch2:Fig1Dynamics}b). 

\begin{figure}
	\centering
	\includegraphics[width=0.95\textwidth]{Ch1Figures/Figure2Times_Thesis_v2.eps}
	\caption{Characteristic times as functions of the Weber number $\Wen$. The times $t_1$, $t_m$, $t_2$, and $t_3$ are normalized by the inertial timescale $\tau_I = D/V$ in panel (a), or by the inertio-capillary timescale $\tau_{\rho\gamma} = (\rho_d D^3/\gamma)^{1/2}$ in panel (b). The black dashed and solid lines represent $t_1 \sim \tau_I$. The black dashed and solid lines represent $t_1 \approx 0.3\tau_I$ and $t_1 \approx 1/6\tau_I$, respectively. The gray dashed lines show  the best straight line fits to the experimental data, $t_m \approx 0.20\tau_{\rho\gamma}$, $t_2 \approx 0.44\tau_{\rho\gamma}$, and $t_3 \approx 0.78\tau_{\rho\gamma}$.}
	\label{Ch2:Fig2Times}
\end{figure}

\section{Weber number dependence of the characteristics times}\label{Ch2:SecTimes}

Next, we look into the dependence of the four different characteristic times on the impact Weber number $\Wen$. The instant $t_1$ of the first peak of the force $F(t)$ scales with the inertial timescale (figure~\ref{Ch2:Fig2Times}a), i.e., $t_1 \sim \tau_I =  D/V$ with different $\Wen$-dependent prefactors ($\approx 0.3$ at low and $\approx 0.167$ at high $\Wen$, respectively). The solid black line in figure~\ref{Ch2:Fig2Times}(a) is the theoretical inertial prediction by \citet{Gordillo2018}, $t_1/\tau_I = 1/6$, and matches our experimental and in particular numerical data. As seen from figure~\ref{Ch2:Fig2Times}, the other three characteristic times scale differently with $\Wen$ than $t_1$. Specifically, $t_2$ and $t_3$ become independent of $\Wen$ when rescaled with the inertial-capillary time $\tau_\gamma = \left(\rho_d D_0^3/\gamma\right)^{1/2}$ while $t_m$ has a weak $\Wen$-dependence at low $\Wen$, and becomes $\Wen$-independent only for $\Wen \gtrsim 10$, see figure~\ref{Ch2:Fig2Times}(b). The reason for this $\Wen$-independent behavior is that the impact process is analogous to one complete drop oscillation \cite{richard2002contact} which is determined by the inertio-capillary time $\tau_{\rho\gamma}$ \cite{rayleigh1879capillary}. Maximum spreading ($t_m$) occurs at almost one-quarter of a full oscillation (consistent with our result $t_m \approx 0.20 \tau_{\rho\gamma}$) whereas the complete contact time $t_3$ takes about one full oscillation (consistent with our result $t_3 \approx 0.78 \tau_{\rho\gamma}$). Finally, the time instant $t_2 \approx 0.44\tau_{\rho\gamma}$ of the second peak in the impact force coincides with the time when the drop's motion changes from being predominantly radial to being vertical, as this moment is associated with the formation of the Worthington jet \cite[p. 18-20]{chantelot2018rebonds}. Note that here for the impact on  the superhydrophobic substrate, the duration of non-zero  forces (e.g.,  for $\Wen = 40$ we find $t_3/\tau_I \approx 5.2$,  figure~\ref{Ch2:Fig1Dynamics}c) is much longer than that for the impact on a hydrophilic surfaces \citet{Gordillo2018, mitchell2019}, where for the same $\Wen = 40$ one has $t_3/\tau_I \approx 2.0$ (also see \S~\ref{Ch2:SecPhobicPhilic}).

\begin{figure}
	\centering
	\includegraphics[width=\textwidth]{Ch1Figures/Figure3_Thesis_v2.eps}
	\caption{Dimensionless peak forces $\tilde{F}_i \equiv F_i/\left(\rho_d V^2D^2\right)$, (a) $\tilde{F}_1$, (b) $\tilde{F}_2$ as functions of $\Wen$. For $\tilde{F}_1$, the black dashed and solid lines represent $\tilde{F}_1 \approx 0.81 + 1.6\Wen^{-1}$ and $\tilde{F}_1 \approx 0.81$, respectively. Using $\tilde{F}_2$, we identify four regimes, I. Capillary ($\Wen < 5.3$), II. Singular jet ($5.3 < \Wen < 12.6$), III. Inertial ($30 < \Wen < 100$), and IV. Splashing ($\Wen > 100$). The black dotted and solid lines represent $\tilde{F}_2 \sim \Wen^{-1}$ and $\tilde{F}_2 \sim \Wen^0$, respectively.  (c) Evolution of the normal force $F(t)$ of an impacting drop for the case with highest $\tilde{F}_2$ ($\Wen = 9$). Note again the outstanding agreement between the experimental and the numerical results, including the various wiggles in the curve, which originate from capillary oscillations. Insets show drop morphology at specific time instants. (d) Snapshots at the instants of the second peak force ($t_2$) for $\Wen =$ (i) $6$, (ii) $12$, (iii) $40$, (iv) $100$, and (v) $300$. (e) Drop geometry at $t_2$ for $\Wen = 40$ (along with the orange drop contour from numerics) to mark its spreading diameter $D_2$, height $h_2$, retraction velocity $v_2$, jet diameter $d_j$ and jet velocity $v_j$. (f) Comparison of the second peak force $\tilde{F}_2$ with its theoretical prediction $\tilde{F}_2^{\text{theory}} = \tilde{v}_j\tilde{v}_2/\tilde D_2$ (equation.~\eqref{Ch2:EqnTheoryForce2}).}
	\label{Ch2:FigMain}
\end{figure}

\section{Influence of Weber number on the first peak}\label{Ch2:SecFirstPeak}

As the drop falls on a substrate, momentum conservation implies $F_1 \sim V(\mathrm{d}m/\mathrm{d}t)$, where the mass flux $\mathrm{d}m/\mathrm{d}t$ can be calculated as $\mathrm{d}m/\mathrm{d}t \sim \rho_d VD^2$ \cite{soto2014}. As a result, $F_1 \sim \rho_d V^2D^2$, as shown in figure~\ref{Ch2:FigMain}(a) for high Weber numbers ($\Wen > 30$, $F_1 \approx 0.81\rho_d V^2D^2$). This asymptote also matches the experimental and theoretical results of similar studies conducted on hydrophilic substrates \cite{zhang2017, Gordillo2018}. Indeed, the first peak force originates from an inertial shock following the impact of drops onto an immobile substrate and is independent of the wettability. Further, the minimum Reynolds number for the current work is $800$, which is well above the criterion ($\Ren > 200$) for viscosity-independent results \cite{zhang2017, Gordillo2018}. One would expect $\tilde{F}_1 \equiv F_1/\rho_d V^2D^2$ to be constant throughout the range of our parameter space. Nonetheless, when $\Wen < 30$, the data deviates from the inertial asymptote. Such deviations have been reported previously on hydrophilic surfaces as well \cite{soto2014}. Here, inertia is not the sole governing force, and it competes with surface tension. We propose a generalization of the first peak of the impact force to $F_1 = \alpha_1\rho_d V^2 D^2 + \alpha_2(\gamma/ D) D^2$, based on dimensional analysis, with $\alpha_1$ and $\alpha_2$ as free parameters. From the best fit to all the experimental and numerical data, we obtain $\tilde{F}_1 \approx 0.81 + 1.6\Wen^{-1}$, which well describes the data, see figure~\ref{Ch2:FigMain}(a).

\section{Influence of Weber number on the second peak}\label{Ch2:SecSecondPeak}

We now focus on the second peak $F_2$ of the impact force $F(t)$. In figure~\ref{Ch2:FigMain}(b), we show the $\Wen$-dependence of the non-dimensional version thereof, $\tilde{F}_2 \equiv F_2/(\rho_d V^2 D^2)$. We identify four main regimes, namely I. Capillary ($\Wen < 5.3$), II. Singular jet ($5.3 < \Wen < 12.6$), III. Inertial ($30 < \Wen < 100$), and IV. Splashing ($\Wen > 100$). The range $12.6 < \Wen < 30$ marks the transition from the singular jet to the inertial regime. 

In regime I ($\Wen < 4.5$), the amplitude $F_2$ of the second peak is smaller than the resolution ($0.5\,\si{\milli\newton}$) of our force transducer, so we cannot distinguish it in experiments. Capillary oscillations dominate the flow in this regime \cite{richard2000bouncing}, leading to more than two peak forces, remarkably perfectly identical to what is observed in our simulations (see figure~\ref{Ch2:FigMain}d-i and supplemental movie~{\color{Myfig} 2}). 

In regime II, with increasing $\Wen$, there is a sharp increase in the amplitude $\tilde{F}_2$ of the second peak. A striking feature of this regime is that the magnitude of the second peak force exceeds that of the first one, $\tilde{F}_2 > \tilde{F}_1$, see figure~\ref{Ch2:FigMain}(c) which  illustrates the case with the highest second peak force ($\tilde{F}_2 = 2.98$, occurring  for $\Wen = 9$, supplemental movie~{\color{Myfig} 3}). The large force amplitude in this regime correlates to the formation of an ultra-thin and high-velocity singular Worthington jet \cite{bartolo2006singular}. Here, the Worthington jet is most pronounced as it results from the collapse of an air-cavity as well as the converging capillary waves (see insets of figures~\ref{Ch2:FigMain}c and~\ref{Ch2:FigMain}d-ii, iii). It is reminiscent of the hydrodynamic singularity that accompanies the bursting of bubbles at liquid-gas interfaces \cite[see \S~{chap:BurstingBubbleVP} and][]{woodcock1953giant, sanjay_lohse_jalaal_2021}. Outside regime II such bubbles do not form, see figure~\ref{Ch2:FigMain}(d-iii -- d-v). Consistent with this view, the case with maximum peak force ($\Wen = 9$, figure~\ref{Ch2:FigMain}b) entrains the largest bubble. Another characteristic feature of this converging flow is that, despite having a small Ohnesorge number ($= 0.0025$) that is often associated with inviscid potential flow inside the drop \cite{molavcek2012quasi}, it still shows high rates of local viscous dissipation near the axis of symmetry (figures~\ref{Ch2:FigMain}c insets and~\ref{Ch2:FigMain}(d-i -- d-iii)), due to the singular character of the flow (also see supplemental movie~{\color{Myfig} 4}).  

When $\Wen$ is further increased, we (locally) find $\tilde{F}_2 \sim \Wen^{-1}$ in the transition regime ($12.6 < \Wen < 30$), followed by $\tilde{F}_2 \sim \Wen^0$ in the inertial regime III ($30 < \Wen < 100$). Specifically, by employing best fits, we obtain

\begin{equation}
	\tilde F_2 = \frac{F_2}{\rho_d V^2 D^2} \approx
	\begin{cases}
		11\Wen^{-1} &(12.6 < \Wen < 30),\\
		0.37 &(30 < \Wen < 100).
	\end{cases}
	\label{Ch2:Eqn1}
\end{equation}

We will now rationalize this experimentally and numerically observed scaling behavior of the amplitude $F_2$ of the second peak using scaling arguments. As already mentioned, figure~\ref{Ch2:FigMain}(d) shows that the second peak in the force at $t_2$ coincides with an upwards jet, which has typical velocity $v_j$ (see \S~\ref{Ch2:SecDetails} for calculation details) and typical diameter $d_j$, figure~\ref{Ch2:FigMain}(e). Figure~\ref{Ch2:FigMain}(d) also illustrates strong radially symmetric flow focusing due to the retracting drop in regimes II and III. We define the recoiling velocity of the drop at time $t_2$ as $v_2$, the droplet height at that moment as $h_2$, and the droplet diameter at that moment as $D_2 = D(t_2)$, see again figure~\ref{Ch2:FigMain}(e). Note that regime II also includes stronger converging capillary waves and the collapsing air cavity (figure~\ref{Ch2:FigMain}c insets and Fig.~\ref{Ch2:FigMain}d-i,ii). The presence of the substrate breaks the symmetry in vertical direction, directing the flow into the Worthington jet. Using continuity and balancing the volume flux at this instant $t_2$, we obtain $v_2  D_2 h_2 \sim v_jd_j^2$. Of course, $D_2$ and $h_2$ are also related by volume conservation. Assuming a pancake-type shape at $t_2$, we obtain $D_2^2h_2 \sim D^3$ \cite{wildeman2016spreading} and therefore, $v_jd_j^2 \sim v_2 D^3/D_2$. As the drop retracts, the velocity of the flow field far away from the jet is parallel to the base (figure~\ref{Ch2:FigMain}d). So, the occurrence and strength of the second peak $F_2$ is mainly a result of the flow opposite to the vertical Worthington jet (figure~\ref{Ch2:FigMain}c-iii -- c-v), which naturally leads to $F_2 \sim \rho_d v_j^2d_j^2$ (momentum flux balance in the vertical direction). Combining the above arguments, we get $F_2 \sim \rho_d v_jv_2 D^3/D_2$ which can be non-dimensionalized with the inertial pressure force $\rho_d V^2D^2$ to obtain

\begin{equation}
	\tilde{F}_2 = \frac{F_2}{\rho_d V^2D^2} \sim \frac{\tilde{v}_j\tilde{v}_2}{\tilde D_2},
	\label{Ch2:EqnTheoryForce2}
\end{equation}

\noindent where, $\tilde{v}_j = v_j/V$, $\tilde{v}_2 = v_2/V$, and $\tilde D_2 = D_2/D$ are the dimensionless jet velocity, drop retraction velocity, and spreading diameter, respectively, all at $t_2$.  

Figure~\ref{Ch2:FigMain}(f) compares the amplitude of the second peak as obtained from the experiments and simulations with the theoretical prediction of equation~\eqref{Ch2:EqnTheoryForce2} (also see \S~\ref{Ch2:SecDetails}). Indeed, this scaling relation reasonably well describes the transitional regime II-III and regime III data. Obviously, in regime I, the theoretical prediction is invalid because the hypothesis of flow focusing breaks down, and capillary oscillations dominate the flow, with no Worthington jet occurring. Further, equation~\eqref{Ch2:EqnTheoryForce2} over-predicts the forces in regime II because efficient capillary waves focusing and air cavity collapse lead to extremely high-velocity singular jets. The entrained air bubble also shields momentum transfer from the singular Worthington jet to the substrate (insets of figure~\ref{Ch2:FigMain}c). 

We finally come to the very large impact velocities of regime IV. Then, when $\Wen \gtrsim 100$, in the experiments splashing occurs \cite{josserand2016drop}, see supplemental movie~{\color{Myfig} 5}. At such high $\Wen$, the surrounding gas atmosphere destabilizes the rim \cite{eggers2010drop, riboux2014experiments}. Therefore, in regime IV, kinetic and surface energies are lost due to the formation of satellite droplets, resulting in diminishing $\tilde{F}_2$ in the experiments (figure~\ref{Ch2:FigMain}b). In contrast, for our axisymmetric (by definition) simulations, the above-mentioned azimuthal instability is absent \cite{eggers2010drop} and the plateau $\tilde{F}_2 \approx 0.37$ continues in this regime. Consequently, equation~\eqref{Ch2:EqnTheoryForce2} holds only for the simulations in regime IV (figure~\ref{Ch2:FigMain}f), and not for the experiments. Further analysis of the experimentally observed fragmentation scenario is beyond the scope of the present work. For future work, we suggest that one could also experimentally probe $F_2$ in this regime by suppressing the azimuthal instability (for instance, by reducing the atmospheric pressure \cite{xu2005drop}). 

\section{Conclusions and outlook}\label{Ch2:SecConclusion}

In this chapter, we have experimentally obtained the normal force profile of water drops impacting superhydrophobic surfaces. To elucidate the physics and study the internal flow, we used direct numerical simulations, which perfectly match the experimental results without any fitting parameter. In the force profile, we identified two prominent peaks. The first peak arises from an inertial shock following the impact of the impacting drop onto the immobile substrate. The hitherto unknown second peak occurs before the drop rebounds. The variation of the amplitude of this peak with Weber number results in four distinct regimes, namely the capillary, singular jet, inertial, and splashing regime. This peak in the force occurs due to the momentum balance when the Worthington jet is created by flow focusing, owing either to capillary waves (singular jet regime) or drop retraction (inertial regime). Surprisingly, even a low Weber number impact (singular jet regime) can lead to a highly enhanced peak in the force profile, triggered by the collapse of an air cavity. Lastly, we have derived scaling relations for these peak forces. Our results thus give a fundamental understanding of the drop impact dynamics on a non-wetting surface and the forces associated with it. Such insight is crucial to develop countermeasures to the failure of superhydrophobicity in technological applications (for e.g., by avoiding the regime $5.3 < \Wen < 12.6$ or reducing the spacing of the textures \cite{lafuma2003}). Interesting and relevant extensions of our work include the study of impact forces of viscous drops, which will show quite different scaling behavior \cite{jha2020viscous}, and of Leidenfrost drops \cite{quere2013leidenfrost}.

\section*{Acknowledgments}
We thank Marie-Jean Thoraval, Uddalok Sen, and Pierre Chantelot for stimulating discussions.


\begin{subappendices}
	
	\section{Experimental method}\label{sec:ExpMethods}

	\begin{sidewaysfigure}
		\centering
		\includegraphics[width=\textwidth]{Ch1Figures/FigureS1Thesis.eps}
		\caption{(a) Experimental setup, consisting of four main units: a drop generation unit, a superhydrophobic substrate, a force measurement unit, and a high-speed photography unit. Relevant appurtenant devices are shown. Also see figure~\ref{Ch2:Fig1Schematic}(a). (b) Snapshot of a water drop sitting on a superhydrophobic substrate with overlaid orange boundary from simulation. The inset shows the scanning electron microscopy (SEM) image of the superhydrophobic surface covered by hydrophobic nanoparticles.}
		\label{Ch2:FigAppSetup}
	\end{sidewaysfigure}

	Figure~\ref{Ch2:FigAppSetup}a illustrates the experimental setup that consists of four main units: a drop generation unit, a superhydrophobic substrate, a force measurement unit and a high-speed photography unit (also see figure~\ref{Ch2:Fig1Schematic}a). 
	
	\subsection{Drop generation unit}
	
	The drop generation unit is used to create drops of different sizes and impact velocities independently (initial drop diameter $D$ ($2.05\,\si{\milli\meter} \le D \le 2.76\,\si{\milli\meter}$) and the impact velocity $V$ ($0.38\,\si{\meter}/\si{\second} \le V \le 2.96\,\si{\meter}/\si{\second}$)). Deionized water drops are created by employing suspended needles that are connected to a syringe pump. To suppress disturbances, the drop is created at a very smooth flow rate, i.e., $0.5\,\si{\milli\liter}/\si{\minute}$. The drop diameter is varied by employing needles of different sizes, and the impact velocity is controlled by changing the distance between the needle and the sample by employing a vertical translation stage.
	
	\subsection{Superhydrophobic substrate}\label{Ch2:sec::substrate}
	A water drop impacts a quartz plate whose upper surface is coated with silanized silica nanobeads with diameter of $20\,\si{\nano\meter}$ (Glaco Mirror Coat Zero; Soft99) \cite{li2017, gauthier2015}  to attain superhydrophobicity (figure~\ref{Ch2:FigAppSetup}b). The advancing and receding contact angles of water drops are $167\si{\degree} \pm 2\si{\degree}$ and $154\si{\degree} \pm 2\si{\degree}$, respectively. 
	
		
	\subsection{Force measurement unit}\label{sec:Sample}
	
	The core apparatuses of the force measurement unit include a sample, a high-precision transducer, a charge amplifier and a data acquisition system. The lower surface of the quartz plate is glued to an aluminum base, which is screwed vertically into a high-precision piezoelectric force transducer (Kistler 9215A) with a resolution of $0.5\,\si{\milli\newton}$. The impact force between the drop and the sample is measured by the collection of the charge generated by the transducer, and the charge is immediately converted into a voltage by employing an amplifier (Kistler 5018A). Comparing to the weak charge signal that is sensitive to the environment noise (such as triboelectricity due to the movement of the cable, or magnetic fields in the environment, etc.), the amplified voltage signal is more robust for transmitting and processing. After that, the amplified analog signals are converted into the digital signals by employing a data acquisition system (NI USB-6361 driven by Labview) at a sampling rate of 100 kHz. Finally, the unit of the measured signal is changed from Voltage ($\si{\volt}$) to Newton ($\si{\newton}$) via the calibration coefficient of the force transducer. 
	
	In our experiment, high-frequency vibrations of the experimental setup are inevitably induced by the drop impact, and they will superimpose on the temporal evolution of the impact force. Based on a well-designed method \cite{li2014}, the influences resulting from these high-frequency vibrations have been successfully removed by employing a low pass filter (with a cut-off frequency of $5\,\si{\kilo\hertz}$).
	
	\subsection{High-speed photography}
	
	Lastly, the fast force sensing technique described above is synchronized with the high-speed photography unit containing a high-speed camera (Photron Fastcam Nova S12) and a micro Nikkor 105 mm f/2.8 imaging lens. To realize a synchronization of the evolution of the drop morphology and the transient force, the high-speed camera is triggered by the data acquisition system when the impact force is larger than $1\,\si{\milli\newton}$. A LED light (CLL-1600TDX) of adjustable output power is used to illuminate the scene of the impingement. We efficiently record the drop impact phenomenon at $10,000$ fps with a shutter speed $1/20,000\,\si{\second}$.
	
	\section{Numerical method}\label{sec:NumMethods}
	This section elucidates the direct numerical simulation framework used to study the drop impact process (figure~\ref{Ch2:Fig1Schematic}b) using the free software program, Basilisk C \cite{basiliskpopinet1}. 
	
	\subsection{Governing equations}
	For an incompressible flow, the mass conservation requires the velocity field to be divergence-free $\left(\boldsymbol{\nabla\cdot v} = 0\right)$.  Furthermore, the momentum conservation  reads 
	
	\begin{align}
		\label{Ch2:EqnNS}
		\frac{\partial\left(\rho\boldsymbol{v}\right)}{\partial t} + \boldsymbol{\nabla\cdot}\left(\rho\boldsymbol{v}\boldsymbol{v}\right) = -\boldsymbol{\nabla} p^{\prime} - [\rho]\left(\boldsymbol{g}\cdot\boldsymbol{z}\right)\boldsymbol{\hat{n}}\delta_s + \boldsymbol{\nabla\cdot}\left(2\eta\boldsymbol{\mathcal{D}}\right) + \boldsymbol{f}_\gamma.
	\end{align}

	\noindent Here, the terms on the left hand side represent momentum advection. On the right hand side, the deformation tensor, $\boldsymbol{\mathcal{D}}$ is the symmetric part of the velocity gradient tensor $\left(\boldsymbol{\mathcal{D}} = \left(\nabla\boldsymbol{v} + \left(\nabla\boldsymbol{v}\right)^{\text{T}}\right)/2\right)$. Further, $p^{\prime}$ denotes reduced pressure field, $p' = p - \rho\boldsymbol{g}\cdot\boldsymbol{z}$, where, $p$ and $\rho\boldsymbol{g}\cdot\boldsymbol{z}$ represent the mechanical and the hydrostatic pressures, respectively, with $\boldsymbol{g}$ and $\boldsymbol{z} = z\boldsymbol{\hat{z}}$ representing the gravitational acceleration and the vertical coordinate vectors, respectively ($z$ is the distance away from the superhydrophobic substrate and $\boldsymbol{\hat{z}}$ is a unit vector, see figure~\ref{Ch2:Fig1Schematic}b). Using this reduced pressure approach ensures an exact hydrostatic balance as described in \citet{popinet2018numerical, basiliskpopinet3}. We also use this reduced pressure approach in chapter~\ref{chap:DropViscousBouncing}. Note that this formulation requires an additional singular body force $\left([\rho]\left(\boldsymbol{g}\cdot\boldsymbol{z}\right)\boldsymbol{\hat{n}}\delta_s\right)$ at the interface. Here, $[\rho]$ is the density jump across the interface, $\boldsymbol{\hat{n}}$ is the interfacial normal vector, $\boldsymbol{\hat{n}} = \boldsymbol{\nabla}H/\|\boldsymbol{\nabla}H\|$, and $\delta_s$ is the \emph{Dirac}-delta function, $\delta_s = \|\boldsymbol{\nabla}H\|$, where $H$ is the Heaviside function. Consequently, $\delta_s$ is non-zero only at the liquid-air interface and has units of $1/\text{length}$ \citet[for detailed derivation, see appendix B of][]{tryggvason2011direct}. Furthermore, we employ one-fluid approximation \cite{prosperetti2009computational, tryggvason2011direct} to solve these equations employing volume of fluid (VoF) method for interface tracking, whereby the Heaviside function can be approximated by the VoF marker function $\Psi$ ($\Psi = 1$ inside the liquid drop, and $\Psi = 0$ in the air). Subsequently, the material properties (such as density $\rho$ and viscosity $\eta$) change depending on which fluid is present at a given spatial location,
	
	\begin{align}
		\label{Ch2::Eqn::density}
		\rho &= \Psi\rho_d + \left(1-\Psi\right)\rho_{a},\\
		\label{Ch2::Eqn::Oh}
		\eta &= \Psi\eta_d + \left(1-\Psi\right)\eta_{a},
	\end{align}

	\noindent where, the subscripts $d$ and $a$ denote drop and air, respectively. The VoF marker function ($\Psi$) follows the advection equation \cite{prosperetti2009computational, tryggvason2011direct}, 
	
	\begin{equation}
		\label{Ch2::Eqn::Vof2}
		\left(\frac{\partial}{\partial t} + \boldsymbol{v\cdot\nabla}\right)\Psi = 0.
	\end{equation}
	
	Lastly, a singular body force $\boldsymbol{f}_\gamma$ is applied at the interfaces to respect the dynamic boundary condition. The approximate forms of this force follows \cite{brackbill1992continuum}
	
	\begin{equation}\label{Ch2::Eqn::SurfaceTension}
		\boldsymbol{f}_\gamma = \gamma\kappa\delta_{s}\boldsymbol{\hat{n}} \approx \gamma\kappa\boldsymbol{\nabla}\Psi.
	\end{equation}

	\noindent Here, $\gamma$ is the drop-air surface tension coefficient and $\kappa$ is the curvature calculated using the height-function method.  During the simulations, the maximum time-step needs to be less than the oscillation period of the smallest wave-length capillary wave because the surface-tension scheme is explicit in time \citep{popinet2009accurate, basiliskpopinet2}.
	
	In our simulations, ideal superhydrophobicity is maintained by assuming that a thin air layer is present between the drop and the substrate \cite{ramirez2020lifting}. The normal force on this substrate can be calculated using \cite{landau2013course}
	
	\begin{equation}
		\label{Ch2:Eqn::force}
		\boldsymbol{F}(t) = \int_\mathcal{A} \left(\left(p-p_0\right)\left(\boldsymbol{I}\boldsymbol{\cdot}\boldsymbol{\hat{z}}\right) - 2\eta_a\left(\boldsymbol{\mathcal{D}}\boldsymbol{\cdot}\boldsymbol{\hat{z}}\right)\right) \mathrm{d}\mathcal{A},
	\end{equation}

	\noindent where, $p$ and $p_0$ are the dynamic pressure distribution at the substrate and the ambient pressure, respectively. Here, $\boldsymbol{I}$ is the second-order identity tensor. Further, $\boldsymbol{\hat{z}}$ is the unit vector normal to the substrate (figure~\ref{Ch2:Fig1Schematic}b) and $\mathcal{A}$ represents substrate's area. Note that the contribution from the second term on the right-hand side of equation~\eqref{Ch2:Eqn::force} is the normal viscous force due to the air layer between the drop and the substrate and is negligible as compared to the pressure integral. Therefore, we can calculate the normal impact force simply by integrating the pressure field at the substrate (see equation~\eqref{Ch2:Eqn::force2}).
	
	Despite a low viscosity associated with the water drops, the viscous dissipation can still be significant in some cases, especially during flow focusing and capillary waves resonance (see figures~\ref{Ch2:FigMain}c,d). To identify these regions of high viscous dissipation, we also measure the viscous dissipation function, given by \cite{landau2013course}
	
	\begin{equation}
		\xi_\eta = 2\eta \left({\boldsymbol{\mathcal{D}}:\boldsymbol{\mathcal{D}}}\right),
		\label{Ch2:Eqn::dissipation1}
	\end{equation}

	\noindent which on non-dimensionalization using the drop diameter ($D$), density ($\rho_d$), and impact velocity ($V$) gives
	
	\begin{equation}
		\tilde{\xi}_\eta \equiv \frac{\xi_\eta}{\rho_dV^3/D} = \frac{2}{Re}\left(\Psi + \frac{\eta_a}{\eta_d}\left(1-\Psi\right)\right)\left(\boldsymbol{\tilde{\mathcal{D}}:\tilde{\mathcal{D}}}\right),
		\label{Ch2:Eqn::dissipation2}
	\end{equation}

	\noindent where, the Reynolds number ($Re = \rho_d VD/\eta_d$) is the ratio of inertial to viscous stresses, and $\boldsymbol{\tilde{\mathcal{D}}} = \boldsymbol{\mathcal{D}}/(V/D)$. 
	
	\subsection{Relevant dimensionless numbers}
	
	In the experiments, the initial drop diameter $D$ ($2.05\,\si{\milli\meter} \le D \le 2.76\,\si{\milli\meter}$) and the impact velocity $V$ ($0.38\,\si{\meter}/\si{\second} \le V \le 2.96\,\si{\meter}/\si{\second}$) are independently controlled. The drop material properties are kept constant (density $\rho_d = 998\,\si{\kilogram}/\si{\meter}^{3}$, surface tension coefficient $\gamma = 73\,\si{\milli\newton}/\si{\meter}$, and dynamic viscosity $\eta_d = 1.0\,\si{\milli\pascal}\si{\second}$). As a result, we identify the following dimensionless numbers,
	
	\begin{align}
		\Wen &= \frac{\rho_dV^2D}{\gamma} \\
		\Ohn &= \frac{\eta_d}{\sqrt{\rho_d\gamma D}}\\
		Bo &= \frac{\rho_dgD^2}{\gamma}
	\end{align}

	where, $\Wen$ is the impact Weber number which is a ratio of the inertial to capillary pressures. The Ohnesorge number ($\Ohn$) is the ratio between the inertio-capillary to the inertio-viscous time scales and is kept constant at $0.0025$ to mimic $2\,\si{\milli\meter}$ diameter water drops. Furthermore, the Bond number ($Bo$) is the ratio of the gravitational to the capillary pressure, which is also fixed at $0.5$ for the same reason. To test the sensitivity of our results on $\Bon$, we also varied its value as $0.0005 \le \Bon \le  0.5$ with no effect on the magnitude of the forces or the four regimes reported in figure~\ref{Ch2:FigMain}. Lastly, to minimize the influence of the surrounding medium, $\rho_a/\rho_d$ and $\eta_a/\eta_d$ are fixed at $10^{-3}$ and $3 \times 10^{-3}$, respectively. 
	
	\subsection{Domain description}
	
	Figure~\ref{Ch2:Fig1Schematic}(b) represents the axi-symmetric computational domain where $r = 0$ denotes the axis of symmetry. A no-slip and non-penetrable boundary condition is applied on the substrate along with zero pressure gradient. Here, we also use $\Psi = 0$ to maintain a thin air layer between the drop and the substrate. Physically, it implies that the minimum thickness of this air layer is $\Delta/2$ throughout the whole simulation duration (where $\Delta$ is the minimum grid size). Further, boundary outflow is applied at the top and side boundaries (tangential stresses, normal velocity gradient, and ambient pressure are set to zero). 
	
	Furthermore, the domain boundaries are far enough not to influence the drop impact process ($\mathcal{L}_{\text{max}} \gg D$). Basilisk C \cite{basiliskpopinet1} also allows for adaptive mesh refinement (AMR) with maximum refinement in the regions of high velocity gradients and at the drop-air interface. With such an adaptive mesh refinement, we can resolve the length scales pertinent to capture the bouncing process, i.e., the flow inside the drop and the region near the substrate. The adaption is based on minimizing the error estimated using the wavelet algorithm \citep{popinet2015quadtree} in the volume of fluid tracers, interfacial curvatures, velocity field, vorticity field and rate of viscous dissipation with tolerances of $10^{-3}$, $10^{-4}$, $10^{-2}$, $10^{-2}$, and $10^{-3}$, respectively  \citep{basiliskvatsaltwopeaks}. We also undertook a mesh independence study to ensure that the results are independent of this mesh resolution. We use a minimum grid size $\Delta = D/1024$ for this study. Note that the cases in regime II (singular jet) requires a refinement of $\Delta \approx D/4098$ near the axis of symmetry. The simulation source codes, as well as the post-processing codes used in the numerical simulations, are permanently available at author's GitHub repository \cite{basiliskvatsaltwopeaks}. 
	
	\subsection{Calculating the time of contact}
	
		\begin{figure}
		\centering
		\includegraphics[width=\linewidth]{Ch1Figures/FigureS_takeOff_Thesis.eps}
		\caption{Temporal variation of the normal contact force for $\Wen = 2$. The insets show two key instances: (i) time $t_3 = 1.14\tau_I$ when $F$ vanishes which marks the contact time of the drop at the substrate and (ii) detachment time ($1.2\tau_I$) as seen from the side view image. Also see supplemental movie~\red{S2}.}
		\label{Fig_TakeOff}
	\end{figure}
	
	In the main text, we emphasize that the time instant $t_3$ at which the normal contact force between the drop and the substrate vanishes is a much better estimate for the drop contact time as compared to the one observed at complete detachment from side view images. This observation is consistent with the literature \cite{bouwhuis2012, van2012direct, lee2020drop,  chantelot_lohse_2021} and is elucidated in Fig.~\ref{Fig_TakeOff} for $\Wen = 2$. In this case, $t_3 = 1.14\tau_I$ whereas the side view images show complete detachment at $t = 1.2\tau_I$. The effect is further enhanced for higher $\Wen$, see for comparisons, supplemental movies~\red{S1} ($\Wen = 40$) and~\red{S2} ($\Wen = 2$).
	
	\section{Superhydrophobic vs. hydrophilic surfaces}\label{Ch2:SecPhobicPhilic}
	
	\begin{figure}
		\includegraphics[width=\textwidth]{Ch1Figures/FigureS2Thesis_v2.eps}
		\caption{Wettability of the surfaces and the corresponding transient force profiles. Water drops depositing on (a) hydrophilic and (b) superhydrophobic surfaces with apparent contact angles of $40\si{\degree} \pm 4\si{\degree}$ and $165\si{\degree} \pm 1\si{\degree}$, respectively. For the superhydrophobic case, the drop boundary from simulation is overlaid in orange. (c) Transient force profiles on the hydrophilic and superhydrophobic surfaces. The initial diameter of the drops is $2.05\,\si{\milli\meter}$, and the impact velocity is $1.20\,\si{\meter}/\si{\second}$, corresponding to $\Wen = 40$. (d) Variation of the first dimensionless peak force $\tilde{F}_1$ as a function of $\Wen$.  Also see figures~\ref{Ch2:Fig1Dynamics}(b) and~\ref{Ch2:FigMain}(a).}
		\label{Ch2:Fig_phobicphilic}
	\end{figure}
	
	To differentiate between impact forces on superhydrophobic surfaces to that of hydrophilic ones \cite{li2014, soto2014, philippi2016, zhang2017, Gordillo2018, mitchell2019, zhang2019}, we carry out test impacts on hydrophilic surfaces. The hydrophilic sample is a quartz plate, cleaned by surfactant, deionized water, alcohol and deionized water in sequence before the experiment. The advancing and receding contact angles of the deionized water drops on the quartz surface are $47\si{\degree} \pm 2\si{\degree}$ and $13\si{\degree} \pm 2\si{\degree}$, respectively (figure~\ref{Ch2:Fig_phobicphilic}a). The superhydrophobic surface is a Glaco-coated quartz plate \cite{li2017, gauthier2015} as described in \S~\ref{Ch2:sec::substrate}, on which the advancing and receding contact angles are $167\si{\degree} \pm 2\si{\degree}$ and $154\si{\degree} \pm 2\si{\degree}$, respectively (figure~\ref{Ch2:Fig_phobicphilic}b). 
	
	Figure~\ref{Ch2:Fig_phobicphilic}c compares the impact on superhydrophobic and hydrophilic substrates for impact corresponding to $\Wen = 40.4$ ($D = 2.05\,\si{\milli\meter}$ and $V = 1.20\,\si{\meter}/\si{\second}$). The comparison shows that in the spreading stage ($0 < t < 2\,\si{\milli\second}$), the transient force profiles overlap. In the time span $2\,\si{\milli\second} < t < 9\,\si{\milli\second}$, the transient force profile of the drop impact on the hydrophilic surface only has slight fluctuations around zero. In contrast to the hydrophilic one, there is an obvious peak force (i.e. $F_2$, corresponding to $t \approx 4.63\,\si{\milli\second}$) in the retraction stage of the drop impact on the superhydrophobic surface.
	
	Furthermore, the impact force $F_1$ on the superhydrophobic surface is equal to the maximum impact force on the hydrophilic surface. To obtain a comprehensive understanding, we extracted experimental data (the maximum impact force) from previous literature performed on hydrophilic surfaces \cite{soto2014, mitchell2016experimental, zhang2017, Gordillo2018, mitchell2019, zhang2019}. Moreover, we carried out experiments on hydrophilic quartz surfaces with an apparent contact angle of $40 \pm 4\si{\degree}$. Then, as shown in Fig.~\ref{Ch2:Fig_phobicphilic}(d), we make a comparison of $F_1$ between previous work (on hydrophilic surfaces) and our work (on both hydrophilic and superhydrophobic surfaces). As shown in Fig.~\ref{Ch2:Fig_phobicphilic}(d), the data on both superhydrophobic and hydrophilic surfaces in our study are consistent with each other. Furthermore, when $We > 30$, the data in the present work and previous literature are consistent with each other. Therefore, $F_1$ only depends on the Weber number, rather than the wettability of the surface. 
	
	\section{Some notes on the different regimes of drop impact}
	\sectionmark{Drop impact regimes}
	\label{Ch2:SecNotes}
	
	\subsection{Regime II: singular Worthington jet}
	
	\begin{figure}
		\centering
		\includegraphics[width=\textwidth]{Ch1Figures/FigureS_RegimeII_thesis1_v2.eps}
		\caption{Evolution of the normal force $\tilde{F}(t) = F(t)/\rho_dV^2D^2$ of a drop impacting on the superhydrophobic surface in the singular Worthington jet regime for $\Wen =$ (a) $9$, (b) $5$, and (c) $12$. Insets in panel (a) show the drop morphology and flow anatomy close to the capillary resonance that leads to a hydrodynamic singularity. Note the outstanding agreement between the experimental (blue line) and the numerical (red line) results, including the various wiggles in the curve, which originate from capillary oscillations for panel (a). The left part of each numerical snapshot shows the dimensionless viscous dissipation function $\tilde{\xi}_\eta$ on a $\log_{10}$ scale and the right part shows the velocity field magnitude normalized with the impact velocity. The black velocity vectors are plotted in the center of mass reference frame of the drop to clearly elucidate the internal flow.}
		\label{Ch2:Fig_RegimeII}
	\end{figure}

	\begin{figure}
		\centering
		\includegraphics[width=\textwidth]{Ch1Figures/FigureS_RegimeII_thesis2_v3.eps}
		\caption{Drop impact on the superhydrophobic surface in the singular Worthington jet regime for $\Wen =$ (a) $9$, (ii) $5$, and (iii) $12$. The left part of each numerical snapshot shows the dimensionless viscous dissipation function $\tilde{\xi}_\eta$ on a $\log_{10}$ scale and the right part shows the velocity field magnitude normalized with the impact velocity. The black velocity vectors are plotted in the center of mass reference frame of the drop to clearly elucidate the internal flow. The numbers in the top right of each numerical snapshot mentions the dimensionless time $t/\tau_I$.}
		\label{Ch2:Fig_RegimeII_2}
	\end{figure}
	
	In the main text, we discussed several features of regime II. In this appendix, we further elucidate this regime using the three representative cases, and look at the transient force profile (figure~\ref{Ch2:Fig_RegimeII}) and the anatomy of flow inside the drops (figure~\ref{Ch2:Fig_RegimeII_2}). We replot the data for $\Wen = 9$ (figures~\ref{Ch2:Fig_RegimeII}a and~\ref{Ch2:Fig_RegimeII_2}a) which shows the maximum force amplitude, and choose $\Wen = 5$ (figures~\ref{Ch2:Fig_RegimeII}b and~\ref{Ch2:Fig_RegimeII_2}b) and $\Wen = 12$ (figures~\ref{Ch2:Fig_RegimeII}c and~\ref{Ch2:Fig_RegimeII_2}c) near the boundaries of regime II (also see supplemental movie~{\color{Myfig} 4}). The transient force profiles show similar features for these three different Weber numbers. After the impact at $t = 0$, there is a sharp increase in the force which reaches the maximum at $t = t_1$. As the drops spread further, their morphology feature distinct pyramidal structures owing to the capillary waves \cite{renardy2003pyramidal} that manifest as oscillations in the temporal evolution of the forces. Then, the drop spreads to a maximum radial extent at $t = t_m$ followed by the retraction phase as the surface tension pulls the drop radially inwards, further enhancing the capillary waves. These traveling capillary waves interact to form an air-cavity, for instance, see $t = 0.9t_2$. The cavity collapses to create high-velocity singular Worthington jets. Subsequently, a bubble is entrained. Comparing the force profile for $\Wen = 9$ with that of $\Wen = 5$ and $\Wen = 12$ reveal differences owing to the corresponding air cavities and bubble entrainment. The flow focusing is the most efficient for $\Wen = 9$, as evidence from the sharp peak in the transient force evolution. This capillary resonance leads to a strong downward momentum jet and hence the maximum amplitude $F_2$ at time $t = t_2$. Bubble entrainment does not occur for either $\Wen = 5$ or $\Wen = 12$ (see $t_2 < t < 1.2t_2$). Consequently, the maximum force amplitude diminishes for these two cases. 
	
	\begin{figure}
		\centering
		\includegraphics[width=\linewidth]{Ch1Figures/FigureS_pressure_Thesis.eps}
		\caption{Contact force and pressure filed during drop impact on the superhydrophobic surface in the singular Worthington jet regime, $\Wen = 9$. (a) Temporal variation of the contact force. Notice that the contact force is negative for $1.1\tau_I \lessapprox t \lessapprox 1.2\tau_I$. (b) Simulation snapshots showing the pressure field $p$ normalized by the inertial pressure $\rho_dV^2$ in the side and bottom view images. The numbers in the top right of each numerical snapshot mentions the dimensionless time $t/\tau_I$.}
		\label{Fig_Pressure}
	\end{figure}
	
	Another characteristic feature of this regime is the occurrence of negative contact force between the drop and the substrate immediately before the formation of a singular Worthington jet and the second peak in normal contact force. Fig.~\ref{Fig_Pressure} illustrates one such case for $\Wen = 9$ where the contact force is negative for $1.1\tau_I \lessapprox t \lessapprox 1.2\tau_I$ implying that the drop is pulling on the substrate instead of pushing it (Fig.~\ref{Fig_Pressure}(a)). Earlier works \cite{grinspan2010impact, li2014, Gordillo2018} have attributed this negative force to the wetting properties of the substrates, particularly adhesion between the drop and the substrate \cite{samuel2011study, liimatainen2017mapping}, viscoelastic effects or deformation of the substrate \cite{Gordillo2018}. However, none of these effects are present in our work. To demystify the occurrence of this negative force, we monitor the pressure field inside the drop (side view, Fig.~\ref{Fig_Pressure}(b)) and on the substrate (bottom view, Fig.~\ref{Fig_Pressure}(b)). We observe large negative pressures (purple regions in the pressure field) on the substrate immediately prior to the formation of a singular Worthington jet and the second peak in normal contact force owing to the negative curvature on the surface of the drop as the air-cavity forms due to focusing of the capillary waves. Consequently, negative capillary pressure causes a pressure deficit inside the drop, and the drop pulls on the substrate instead of pushing it (brown regions in the pressure field).
	
	\subsection{Transitional regime II-III and inertial regime III}
	
	\begin{sidewaysfigure}
		\centering
		\includegraphics[width=180mm]{Ch1Figures/FigureS_We30TransitionThesis_v2.eps}
		\caption{Snapshots of drop shape with time at different Weber numbers, $\Wen = $  (a) 20 (b) 80.}
		\label{Ch2:Fig_Transition}
	\end{sidewaysfigure}
	
	In the main text, we used the flow focusing due to drop retraction to find an expression for the amplitude $F_2$ (see, equation~\eqref{Ch2:EqnTheoryForce2}) that entails two scaling behaviors depending on the $\Wen$ (see equation~\eqref{Ch2:Eqn1}). To address the crossover of these two scaling relations, (i.e., $\Wen = 30$), we check the deformation of the drop at the moment of maximum spreading and the corresponding position of the drop apex. 
	
	To make a comparison, we exemplary choose $\Wen = 20$ ($D = 2.05\,\si{\milli\meter}, V = 0.83\,\si{\meter}/\si{\second}$) and $\Wen = 80$ ($D = 2.05\,\si{\milli\meter}, V = 1.69\,\si{\meter}/\si{\second}$), and show their impact behaviors in figures~\ref{Ch2:Fig_Transition}(a) and (b), respectively. By simulations, the anatomy of the inner flow field of the drop are discernible (see the right panels). For the case $\Wen = 20$, the solid-liquid contact region is close to the initial drop diameter when $F_1$ is attained at $t_1 = 0.6\,\si{\milli\second}$. Meanwhile, the excited capillary wave propagates along the drop surface and then deforms the drop into a pyramidal shape at $1.5\,\si{\milli\second}$. Then, the drop reaches its maximum spreading diameter $D_m$ at $t_m = 2.3\,\si{\milli\second}$. Notice that at this moment, the drop apex is higher than the height of the rim and is still moving downwards. After that, the drop starts to recoil, and the drop apex descends to its lowest level after $t_m$. During the recoil, the retreating drop deforms into a pancake shape with air in the center, as shown at $4.0\,\si{\milli\second}$. As time progresses, the retracting flow fills the cavity and creates an upward jet at $t_2 = 4.5\,\si{\milli\second}$, which results in $F_2$.
	
	On the other hand, for the case with $\Wen = 80$, a thin liquid film appears, and the solid-liquid contact area is close to the initial drop diameter when $F_1$ is attained, similar to the case with $\Wen = 20$. However, unlike the previous case, $t_1 = 0.2\,\si{\milli\second}$ (see \S~\ref{Ch2:SecTimes}). Moreover, there is no obvious capillary wave propagating on the drop surface, as shown at $1.0\,\si{\milli\second}$. Then, the drop apex continuously moves downwards, and its height reaches the height of the rim, and this moment happens before the drop reaches its maximum spreading diameter $D_m$ at $t_m = 2.0\,\si{\milli\second}$. Shortly after $t_m$, the drop recoils, while the film thickness in the central region remains the same (see $4.5\,\si{\milli\second}$, \cite{eggers2010drop}). Then, the thickening retracting flow converges and collides at the film center to form an upward jet to result in $F_2$, as shown at $4.9\,\si{\milli\second}$.
	
	Based on the above results, we gain the following insight. For small $\Wen$ (figure~\ref{Ch2:Fig_Transition}a, $\Wen = 20$), the drop attains $D_{m}$ (at time $t_m$) before its apex descends to its lowest level (at time $\sim$ $D/V$), leading to a puddle-shaped drop \cite{wildeman2016spreading}. This observation indicates that at $t_2$, a competition exists between two flows in the central region of the drop, respectively, coming from the rim and the drop apex. However, for large $\Wen$ (figure~\ref{Ch2:Fig_Transition}b, $\Wen = 80$), the drop apex attains its lowest level before $t_m$, so the drop has a pizza shape \cite{eggers2010drop, wildeman2016spreading}. Equating the two timescales together, one obtains a crossover Weber number $\Wen^* = 25$, which is close to the value $30$ observed in our work. Alternatively, equating the two scaling relations in equation~\eqref{Ch2:Eqn1} gives a more accurate estimate of the crossover Weber number as $\Wen^* = 29.7$.
	
	\begin{sidewaysfigure}
		\includegraphics[width=180mm]{Ch1Figures/FigureS_SplashingThesis.eps}
		\caption{Drop impact on the superhydrophobic surface at a high Weber number, $\Wen = 225$. (a) Evolution of the transient impact force. (b) Snapshots of the corresponding drop geometry in the spreading and recoiling stages. Notice that the experimental and numerical $F(t)$ magnitudes only disagree near $t = t_2$. Further, the time at which the second peak is reached is still at $t_2 \approx 0.44\tau_{\rho\gamma}$, as explained in the main text. The left part of each numerical snapshot shows the dimensionless viscous dissipation function, $\tilde{\xi}_\eta$ on a $\log_{10}$ scale and the right part shows the velocity field magnitude normalized with the impact velocity, $\|\boldsymbol{v}\|/V$. The black velocity vectors are plotted in the center of mass reference frame of the drop to clearly elucidate the internal flow.}
		\label{Ch2:Fig_Splashing}
	\end{sidewaysfigure}
	
	\subsection{Regime IV: drop splashing}

	The main text reported differences between the experimental and numerical observations in regime IV. Here, we further delve into this discrepancy to identify the reasons behind it (figure~\ref{Ch2:Fig_Splashing}). At high impact velocities ($\Wen = 225$ in figure~\ref{Ch2:Fig_Splashing}), splashing occurs in the experiments \cite{josserand2016drop}. At such high $\Wen$, the surrounding gas atmosphere destabilizes the rim, breaking it  \cite{eggers2010drop, riboux2014experiments}. The limit for splashing observed in this work ($\Wen \ge 100$, see figure~\ref{Ch2:FigMain}) is in agreement with the predictions in previously predicted works of \citet{derby2010inkjet, riboux2014experiments, lohse2022fundamental} (see figure~\ref{Ch0::Fig4}).  
	
	In regime IV, a part of kinetic and surface energies are lost due to the formation of satellite drops (figure~\ref{Ch2:Fig_Splashing}b), resulting in diminishing $\tilde{F}_2$ in the experiments (Fig~\ref{Ch2:Fig_Splashing}(a)). Obviously, such azimuthal instability is absent in the simulations (axisymmetric by definition), which leads to a better flow-focusing at the center. Indeed, \citet{eggers2010drop} were able to simulate cases with $\Wen$ as high as $1000$ without breakup. Consequently, equation~\eqref{Ch2:EqnTheoryForce2} holds only for the simulations in regime IV and not for the experiments. Notice that the experimental and numerical $F(t)$ magnitudes only disagree near $t = t_2$. Further, the time at which the second peak is reached is still at $t_2 \approx 0.44\tau_{\rho\gamma}$, as explained in the main text. For future work, we suggest that one could also experimentally probe $F_2$ in this regime by suppressing the azimuthal instability (for instance, by reducing the atmospheric pressure \cite{xu2005drop}). 
	
	\pagebreak 
	
	\section{Calculation of the jet and retraction characteristics}\label{Ch2:SecDetails}
	
	\begin{figure}
		\includegraphics[width=\textwidth]{Ch1Figures/FigureS_ExpU2Uj_Thesis_v2.eps}
		\caption{Experimental time evolution of (a) spread radius $\mathcal{R}(t)$ and (b) the drop height $\mathcal{H}(t)$. Inset illustrates the drop geometry. The retraction velocity $v_2 = \dot{\mathcal{R}}(t_2)$ and the jet velocity $v_j = \dot{\mathcal{H}}(t_2)$ are represented by the slopes of the solid lines at $t_2$ in (a) and (b), respectively. Here, $\Wen = 40$.}
		\label{Ch2:Fig_ExpU2Uj}
	\end{figure}
	
	\subsection{Experiments}	
	In this section, we illustrate how to extract $v_2$ and $v_j$ from the experiments. As an example, we choose $\Wen = 40$. As shown in figure~\ref{Ch2:Fig_ExpU2Uj}, we first track the instantaneous values of the width $2\mathcal{R}(t)$ (figure~\ref{Ch2:Fig_ExpU2Uj}a) and the height $\mathcal{H}(t)$ (figure~\ref{Ch2:Fig_ExpU2Uj}b) of the drop. We observe that just after the impingement, the drop height decreases with a constant velocity \cite{eggers2010drop, Gordillo2018} until the inertial shock propagates throughout the drop. As time progresses, $\mathcal{R}(t)$ and $\mathcal{H}(t)$ respectively reach their maximum and minimum values simultaneously (at $t_{\mathrm{m}} \approx 2.5\,\si{\milli\second}$). Moving forward in time, $\mathcal{R}(t)$ decreases, whereas $\mathcal{H}(t)$ increases linearly until $t_2$. After this moment, we observed a sharp increase of $\mathcal{H}(t)$ until the drop breaks into the base drop and a small droplet (see the insets in figure~\ref{Ch2:Fig1Dynamics}a of the main text). We define the recoiling velocity $v_2$ of the drop and the jet velocity $v_j$ as:
	
	\begin{align}
		v_2 = \dot{\mathcal{R}}(t_2) &= \left. \frac{\mathrm{d}\mathcal{R}(t)}{\mathrm{d}t} \right|_{t_2},\\
		v_j = \dot{\mathcal{H}}(t_2) &= \left. \frac{\mathrm{d}\mathcal{H}(t)}{\mathrm{d}t} \right|_{t_2}.
	\end{align}
	
	As shown in figure~\ref{Ch2:Fig_ExpU2Uj}(a), $v_2= \dot{\mathcal{R}}(t_2)$ is obtained by a linear fitting (black line) to the experimental data around $(t_2, \mathcal{R}(t_2))$. Similarly, as shown in figure~\ref{Ch2:Fig_ExpU2Uj}(b), $v_j = \dot{\mathcal{H}}(t_2)$ is obtained by a linear fitting (black line) to the experimental data around $(t_2, \mathcal{H}(t_2))$. Note that in the experiments, we can only measure the maximum height of the drop. Consequently, when the rim thickness exceeds the drop's height, $\mathcal{H}$ identifies the height of the rim (particularly for $t < t_m$). So, we use the datapoints after $t = t_2$ to calculate the jet velocity. Nonetheless, the jet velocity $v_j$ extracted at $t = t_2^{+}$ from the experiments are consistent very well with our simulation (where we can precisely calculate the jet velocity, see \S~\ref{sec::calculations Sim}), as well as the results obtained by \citet{bartolo2006singular}, as discussed in \S~\ref{sec::calculations results}.
	
	
	\subsection{Simulations}
	\label{sec::calculations Sim}
	
	\begin{sidewaysfigure}
		\includegraphics[width=180mm]{Ch1Figures/FigureS_calculationsThesis_v2.eps}
		\caption{Calculation of the (i) jet velocity ($v_j$) and (ii) retraction velocity ($v_2$) for Eq.~(\red{2}) of the main manuscript for two representative cases: $\Wen =$ (a) $9$ and (b) $100$. Inset illustrates the drop geometry where $\mathcal{H}$ is the height of the drop at the axis of symmetry and $2\mathcal{R}$ is its radial extent. The jet velocity is $v_j = \dot{\mathcal{H}}(t_2)$ and the retraction velocity is $v_2 = \dot{\mathcal{R}}(t_2)$. Notice that the time at which second peak is reached still scales with the inertio-capillary timescale $t_2 \sim \tau_{\rho\gamma}$, as described in the main text, irrespective of the $\Wen$ ($t_2 = 0.405\tau_{\rho\gamma}$ for $\Wen = 9$, and $t_2 = 0.437\tau_{\rho\gamma}$ for $\Wen = 100$).}
		\label{Ch2:Fig_calculations}
	\end{sidewaysfigure}

	To characterize the jet, we track the interfacial location (or height of the drop, $\mathcal{H}(t)$) at the axis of symmetry ($r = 0$). Similarly, to characterize retraction, we track the radial extent of the drop ($2\mathcal{R}(t)$). Further, $\dot{\mathcal{H}} = \mathrm{d}\mathcal{H}/\mathrm{d}t$ measures the velocity of this jet, and $\dot{\mathcal{R}} = \mathrm{d}\mathcal{R}/\mathrm{d}t$ accounts for the retraction velocity. Figure~\ref{Ch2:Fig_calculations} shows the temporal variation of $\dot{\mathcal{H}}$ (panel i) and $\dot{\mathcal{R}}$ (panel ii) for two representative Weber numbers, $\Wen = 9$ (panel a) and $100$ (panel b). As the drop impacts, the top of the drop keeps moving with a constant velocity ($\dot{\mathcal{H}} \approx V$) \cite{eggers2010drop, Gordillo2018}, consistent with our experiments. However, during this period, the radial velocity magnitude increases to a maximum and then decreases to zero at the instant of maximum spreading. 
	
	For low to moderate Weber number impacts ($\Wen = 9$ in figure~\ref{Ch2:Fig_calculations}a), the pyramidal morphology result in capillary oscillations aiding the flow focusing in the retraction phase. Consequently, both the normal force ($F$) and the jet velocity ($\dot{\mathcal{H}}$, figure~\ref{Ch2:Fig_calculations}a-i) reach the maxima simultaneously at $t = t_2$. Further, the retraction velocity (figure~\ref{Ch2:Fig_calculations}a-ii) show oscillations due to capillary waves.
	
	On the other hand, for high Weber number impacts ($\Wen = 100$ in figure~\ref{Ch2:Fig_calculations}b), the jet velocity is minimum at the instant of maximum spreading. Then, Taylor-Culick type retraction occurs increasing the retraction velocity to a maximum which then decreases due to finite size of the drop \cite{bartolo2005retraction, eggers2010drop, pierson2020revisiting, deka2020revisiting}.  During this retraction phase, flow focusing and asymmetry provided by the substrate lead to a sudden increase in the jet velocity (figure~\ref{Ch2:Fig_calculations}b-i) that is immediately followed by occurrence of the second peak in the transient force profile (at $t_2$). The retraction velocity at this instant is very close to its maximum temporal value (figure~\ref{Ch2:Fig_calculations}b-ii). 
	
	For both cases, notice that the time at which second peak is reached still scales with the inertio-capillary timescale $t_2 \sim \tau_{\rho\gamma}$, as described in the main text, irrespective of the $\Wen$ ($t_2 = 0.405\tau_{\rho\gamma}$ for $\Wen = 9$, and $t_2 = 0.437\tau_{\rho\gamma}$ for $\Wen = 100$).
	
	In summary, 
	\begin{align}
		v_j &= \dot{\mathcal{H}}(t_2)\\
		v_2 &=  \dot{\mathcal{R}}(t_2).
	\end{align}
	
	Lastly, we can also characterize the maximum lateral extent $D_2$ of the drop at the instant $t_2$ of second peak in the normal reaction force $F_2$ as
	
	\begin{align}
			D_2 = 2\mathcal{R}\left(t_2\right),
	\end{align}
	
	\noindent in both experiments as well as simulations.
	
	\subsection{Results}
	\label{sec::calculations results}
	
	\begin{figure}
		\includegraphics[width=\textwidth]{Ch1Figures/FigureS_detailsInternalFlowThesis_v2.eps}
		\caption{(a) Drop geometry at $t_2$ for $\Wen = 40$ (along with the drop contour from numerics in orange) to illustrate the drop spreading diameter $D_2$, drop height $h_2$, retraction velocity $v_2$, jet diameter $d_j$ and jet velocity $v_j$. (b) Variation of the dimensionless spreading diameter at $t_m$ and $t_2$ (given by $\tilde D_{max} = \tilde D (t_m)  $ and $\tilde D_2 = \tilde D(t_2) $, respectively). The gray dotted and solid lines represent $\tilde D_{max} \sim \Wen^{1/4}$ and $\tilde D_2 \approx 1.4$, respectively. (c) Variation of drop retraction velocity $\tilde{v}_2$ at $t_2$ with $\Wen$. The gray solid line represents $v_2 \sim V$. The gray dotted and dashed-dotted lines correspond to $\Wen^{-1/2}$ and $\Wen^{-1/4}$, respectively,  and are meant only as guides to the eye. (d) Variation of the (non-dimensionalized) jet velocity $\tilde{v}_j = v_j/V$ with $\Wen$. The data from \citet{bartolo2006singular} are also shown in the same panel. The gray solid line represents $\tilde{v}_j \sim \Wen^{-1}$. The gray dotted and dashed-dotted lines correspond to $\Wen^{1/2}$ and $\Wen^{1/4}$, respectively, and are meant only as guides for the eye.}
		\label{Ch2:Fig_details}
	\end{figure}

	We will devote the rest of this appendix to relate the different flow properties in equation~\eqref{Ch2:EqnTheoryForce2} (also see figure~\ref{Ch2:Fig_details}a) to the control parameter, i.e., the impact Weber number $\Wen$. For the transitional regime II-III ($12.6 < \Wen < 30$), at the moment of second peak, the dimensionless diameter ($\tilde{D}_2$) and the dimensionless drop retraction velocity ($\tilde{v}_2$) are independent of the impact Weber number $\Wen$ (figures~\ref{Ch2:Fig_details}b, c). Further, the jet velocity decreases with increasing Weber number following $\tilde{v}_j = v_j/V \sim 1/\Wen$ (figure~\ref{Ch2:Fig_details}d). This decrease is consistent with the data extracted from \citet{bartolo2006singular}. Substituting these in equation~\eqref{Ch2:EqnTheoryForce2}, one obtains $\tilde{F}_2 \sim 1/\Wen$. However, the prefactor that best fits the experimental and numerical data in equation~\eqref{Ch2:Eqn1} is much larger than order 1, which may be caused by the enhanced flow and momentum focusing due to both capillary waves and drop retraction.
	
	For regime III ($30 < \Wen < 100$), there is a slight increase in $\tilde{D}_2$ (figure~\ref{Ch2:Fig_details}b) but it is still best represented by a plateau. Furthermore, with increasing $\Wen$, $\tilde{v}_2$ decreases whereas $\tilde{v}_j$ increases (figures~\ref{Ch2:Fig_details}c,d). We have shown gray lines as guides to the eye to represent these trends. However, due to limited range of $\Wen$, we refrain from claiming any scaling relations here. Coincidentally, these changes in $\tilde{D}_2$, $\tilde{v}_2$, and $\tilde{v}_j$ compensate each other such that equation~\eqref{Ch2:EqnTheoryForce2} still holds. Consequently, the second peak force scales with the inertial pressure force $F_2 \sim \rho_d V^2D^2$ (equation~\eqref{Ch2:Eqn1}).
	
	Alternatively, we can use the expressions for the amplitude of the second peak of force between the drop and the substrate to predict the velocity of the Worthington jet. For $\Wen \gg 1$, the drop forms a thin sheet at the instant of maximum spreading ($t = t_m$, see figures~\ref{Ch2:Fig_Transition}b and~\ref{Ch2:Fig_Splashing}). This sheet retracts following Taylor-Culick type retraction at low Ohnesorge numbers ($\Ohn \ll 1$). As a result, the retraction velocity scale can be given as \cite{bartolo2005retraction, eggers2010drop}:
	
	\begin{align}
		v_2 \sim \sqrt{\frac{\gamma}{\rho_d h_2}},
		\label{Eqn::TC}
	\end{align}
	
	\noindent where, $\gamma$ and $\rho_d$ are the surface tension coefficient and density of the liquid drop, respectively. Further, $v_2$ is the retraction velocity at $t = t_2$ and $h_2$  is the height of the drop at that instant which is related to the spreading diameter following volume conservation as $h_2 \sim D^3/D_2^2$. Substituting this expression in equation~\eqref{Eqn::TC} and normalizing with $V$, we get
	
	\begin{align}
		\tilde{v}_2 \sim D_2\sqrt{\frac{\gamma}{\rho_dV^2 D^3}} = \frac{\tilde{D}_2}{\sqrt{\Wen}}.
		\label{Eqn::TC2}
	\end{align}
	
	\noindent The finite size of the retracting drop may account for the deviations from equation~\eqref{Eqn::TC2} in figures~\ref{Ch2:Fig_details}(b,c) \cite{pierson2020revisiting, deka2020revisiting}. Further, using $\tilde{F}_2 \sim \tilde{v}_2\tilde{v}_j/\tilde{D}_2 \sim \mathcal{O}\left(1\right)$ for $\Wen \gg 1$, we obtain
	
	\begin{align}
		\tilde{v}_j \sim \sqrt{\Wen}.
		\label{Eqn::TCj}
	\end{align}
	
	\noindent Unfortunately, we cannot confirm the validity of this scaling behavior in figure~\ref{Ch2:Fig_details}(d) due to a limited range of $\Wen$ as mentioned above. For completeness, in figure~\ref{Ch2:Fig_details}(b), we also show the maximum spreading diameter from our experiments and simulations are in a remarkable agreement. We refer the readers to \cite{clanet2004, eggers2010drop, laan2014maximum} for further discussions on the influence of $\Wen$ on the maximum spreading diameter.
		
	\subsection{Outlook on the scaling relations} 
	
	In this section and figure~\ref{Ch2:Fig_details}, we probe several scaling behaviors in an attempt to relate the internal flow characteristics, i.e., the jet and retraction velocities, to the impact Weber number. However, verifying the predicted scaling behaviors requires a larger range of Weber numbers that we do not study due to experimental and numerical limitations. For example, at very high-velocity impacts, the drop splashes and breaks into many satellite droplets in the experiments \cite{riboux2014experiments}. For future work, we suggest that one could experimentally probe this regime by suppressing the azimuthal instability (for instance, by reducing the atmospheric pressure \cite{xu2005drop}). However, even in such a scenario, at very high impact velocities ($\Wen \gg 1$), the substrate roughness may play a role in both drop spreading and retraction \cite{visser2015dynamics}. In numerical simulations, one can probe this regime by using drops that are slightly more viscous than water, as done by \citet{eggers2010drop}. However, such a study is numerically costly for water drops impacting at very large $\Wen$ due to the separation of length scales between the initial diameter of the drop and the very thin lamella during spreading. Furthermore, the interfacial undulations (traveling capillary waves) further restrict both the spatial and temporal resolutions.
	
	\section{Code availability}
	
	All codes used in this chapter are permanently available at \citet{basiliskvatsaltwopeaks}.
	
	\section{Supplemental movies}
	These supplemental movies are available at \citet[\href{https://youtube.com/playlist?list=PLf5C5HCrvhLGmlYTF1Gg2WviZ-Bkmy2qr}{external YouTube link,}][]{vatsalDropForcessuppl}. 
	
	\begin{enumerate}
		\item[SM1:] The evolution of the transient force of a water drop impacting on the superhydrophobic surface at a moderate Weber number $\Wen = 40$ (corresponding to $D = 2.05\,\si{\milli\meter}$ and $V = 1.20\,\si{\meter}/\si{\second}$), with simultaneous drop geometry recorded experimentally at 10,000 fps with the exposure time of 1/20,000 s. The left part of the numerical video shows the dimensionless viscous dissipation function on a $\log_{10}$ scale and the right part shows the velocity field magnitude normalized with the impact velocity.
		\item[SM2:] The evolution of the transient force of a water drop impacting on the superhydrophobic surface at a low Weber number $\Wen = 2$ (Regime I: capillary), with simultaneous drop geometry evolution. The left part of the numerical video shows the dimensionless viscous dissipation function on a $\log_{10}$ scale and the right part shows the velocity field magnitude normalized with the impact velocity.
		\item[SM3:] The evolution of the transient force of a water drop impacting on the superhydrophobic surface at Weber number $\Wen = 9$ (corresponding to $D = 2.76\,\si{\milli\meter}$ and $V = 0.49\,\si{\meter}/\si{\second}$), with simultaneous drop geometry recorded experimentally at 10,000 fps with the exposure time of 1/50,000 s. The ultra-thin and fast singular jet is reminiscent of the hydrodynamic singularity. The left part of the numerical video shows the dimensionless viscous dissipation function on a $\log_{10}$ scale and the right part shows the velocity field magnitude normalized with the impact velocity.
		\item[SM4:] Boundaries of the singular jet regime: the evolution of the transient force of a water drop impacting on the superhydrophobic surface at three representative Weber numbers in regime II, $\Wen = 5$, $\Wen = 9$, and $\Wen = 12$, with simultaneous drop geometry evolution. The left part of the numerical video shows the dimensionless viscous dissipation function on a $\log_{10}$ scale and the right part shows the velocity field magnitude normalized with the impact velocity.
		\item[SM5:] Contact force and pressure filed during drop impact on the superhydrophobic surface in the singular Worthington jet regime, $\Wen = 9$. Simulation shows the pressure field $p$ normalized by the inertial pressure $\rho_dV^2$ on the left and the magnitude of velocity field $\|\boldsymbol{v}\|$ normalized by the impact velocity $V$ on the right in the side view images and pressure field in the bottom view images.
		\item[SM6:] The evolution of the transient force of a water drop impacting on the superhydrophobic surface at a high Weber number $\Wen = 225$ (drop splashing regime, corresponding to $D = 2.05\,\si{\milli\meter}$ and $V = 2.83\,\si{\meter}/\si{\second}$), with simultaneous drop geometry recorded experimentally at 10,000 fps with the exposure time of 1/20,000 s. The left part of the numerical video shows the dimensionless viscous dissipation function on a $\log_{10}$ scale and the right part shows the velocity field magnitude normalized with the impact velocity.
		\item[Bonus:] Conference (Physics@Veldhoven 2022) talk titled \lq\lq How much force is required to play ping-pong with water droplets?\rq\rq\, presenting the results from this chapter. 
	\end{enumerate}

	\begin{figure*}
		\centering
		\includegraphics[width=\textwidth]{Ch1Figures/QRcodesChapter1.eps}
	\end{figure*}

\end{subappendices}