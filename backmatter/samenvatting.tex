\chapter[Samenvatting]{Samenvatting\raisebox{.3\baselineskip}{\normalsize\footnotemark}}
\footnotetext{I would like to thank Sander Huisman, Maaike Rump, and Carola Seyfert for proofreading the Dutch summary.}

%\chapter[Samenvatting]{Samenvatting}

In dit proefschrift hebben wij verschillende verschijnselen aan het vrije oppervlak onderzocht om de rol van viskeuze spanningen te illustreren. Deel~\ref{PartA} (hoofdstukken~\ref{chap:DropForces}--\ref{chap:DropOnDrop}) bestudeerde de impact van sferische vloeistofdruppels op niet-natte substraten. Deel~\ref{PartB} (hoofdstukken~\ref{chap:TaylorCulick}--\ref{chap:BurstingBubbleVP}) concentreerde zich op capillair-gedreven terugtrekking van films en het barsten van vrije-oppervlakte bellen.

In \textbf{hoofdstuk~\ref{chap:DropForces}} bestuderen we waterdruppels die inslaan op niet-natte substraten. We vinden dat zowel de impact als de take-off gepaard gaan met een toename van de impactkracht op het substraat. De traagheidsdrukkracht bepaalt de grootte van deze beide pieken in de normale reactiekracht. Maar verrassend genoeg kunnen zelfs botsingen bij lage snelheid leiden tot een opmerkelijk hoge tweede piek in de normaalkracht, die zelfs groter kan zijn (bijna driemaal) dan de eerste. Deze verbetering kan worden toegeschreven aan de ineenstorting van een luchtholte binnenin de vloeistofdruppel, wat leidt tot enkelvoudige Worthington-stralen.Onze resultaten geven dus een fundamenteel inzicht in de dynamica van de druppelinslag op een niet-bevochtigd oppervlak en de krachten die ermee gepaard gaan. Zulk inzicht is van cruciaal belang om tegenmaatregelen te ontwikkelen tegen het falen van superhydrofobiciteit in technologische toepassingen (bijvoorbeeld door het vermijden van het regime met hoge impactkrachten).

In \textbf{hoofdstuk~\ref{chap:DropViscousBouncing}} we bepalen de overgang van stuiterende naar niet-stuite\-ren\-de druppels die op een niet-nat substraat vallen. Tijdens het druppel impact proces verstoort de viskeuze dissipatie het interne momentum. Een druppel zal ophouden met stuiteren en op het substraat blijven liggen als zijn opwaartse momentum (aangedreven door capillariteit en tegengewerkt door viskeuze spanningen)  na het proces van druppelinslag, verspreiding en terugtrekking onvoldoende is om de zwaartekracht te overwinnen. Zwaar\-tek\-racht en viscositeit werken dus samen om te voorkomen dat druppels van een niet-nat substraat stuiteren. Wij stellen verder vast dat dicht bij deze overgang het terugslagproces onafhankelijk is van de inslagparameters. Deze waarneming ontkoppelt de latere fasen van de terugslag van de initi{\"e}le dynamica van de inslag. Deze resultaten zijn nuttig in toepassingen waar het terugkaatsen van een druppel moet worden onderdrukt, bijvoorbeeld bij inkjetdruk, koeltoepassingen, de toepassing van pesticiden en criminele forensische wetenschap.

In \textbf{hoofdstuk~\ref{chap:DropBouncingOnFilm}} onderzoeken we druppels die weerkaatsen op viskeuze vloeistoffilms die atomair gladde substraten nabootsen. Het afstotend gedrag van dergelijke substraten vereist de aanwezigheid van een luchtlaag opgesloten tussen de stotende druppel en de vloeibare film. Druppels die inslaan op viskeuze vloeistof films vertonen twee verschillende stuiteren regimes, namelijk het substraat-onafhankelijke en het substraat-afhankelijke stuiteren. In het eerste regime wordt de botsingsdynamiek niet be{\"i}nvloed door de aanwezigheid van de viskeuze film, als gevolg van zijn hoge viscositeit of verwaarloosbare dikheid. In het laatste echter beïnvloeden zowel de eigenschappen van de druppel als die van de film de terugstuitdynamiek. Binnen de substraatonafhankelijke grens wordt de afstoting onderdrukt zodra de druppelviscositeit een kritische waarde overschrijdt, zoals bij superamphiphobische substraten die in hoofdstuk~\ref{chap:DropViscousBouncing}. Het substraat-afhankelijke regime laat ook een grens toe voor druppels met lage viscositeit, waarin alleen de eigenschappen van de film de remming van afstoting bepalen.

In \textbf{hoofdstuk~\ref{chap:DropOnDrop}} hebben wij ontdekt dat in aanwezigheid van een niet-bevochtigend substraat, de druppel-op-druppel botsing vijf terugslagscenario's oplevert, waarvan er vier geen coalescentie inhouden. De botsende druppel tilt in twee van de vier terugspringscenario's een luie sessiele op. Als er voldoende energie tussen de druppels wordt overgedragen, kunnen beide druppels van het substraat kunnen nemen, terwijl in sommige gevallen de botsende druppel de sessiele druppel van het substraat schopt, maar zelf niet kan stuiteren. Bovendien illustreert een {\'e}{\'e}n-op-{\'e}{\'e}n vergelijking tussen de experimenteel en numeriek bepaalde druppelgrenzen en de mechanische energieën van het massamiddelpunt de kracht van de directe numerieke simulaties voor het kwantitatief voorspellen van de dynamica van druppel-op-druppel inslag. Inzichten uit de hoofdstukken~\ref{chap:DropBouncingOnFilm}  en~\ref{chap:DropOnDrop}  zijn belangrijk omdat voor de meeste industriële toepassingen, zoals inkjet printing of additive manufacturing, druppels worden afgezet op een reeds bestaande laag van een andere druppel of film. Vandaar dat de relatieve precisie van de druppelafzetting en zijn vormevolutie het succes of het falen van deze toestellen kunnen bepalen. 

Merk op dat, hoewel wij directe numerieke simulaties doen van de verschil- lende scenario’s van druppelinslagen in de hoofdstukken~\ref{chap:DropForces}--\ref{chap:DropOnDrop}, wij dat doen met behulp van de continu{\"u}mvergelijkingen. Bijgevolg is onze numerieke methode ontoereikend om te voorspellen of een druppel al dan niet met een andere druppel of met het substraat zal samensmelten. Wij halen deze informatie uit de experimenten. Het al of niet samensmelten van grensvlakken hangt namelijk af van verschillende multifysische aspecten, waaronder oppervlakte- asperiteiten en vanderwaalskrachten. Voorspelling van zulke coalescentie (of breuk) gedragingen valt buiten het bereik van dit schrift, maar men kan een consistente moleculaire dynamica techniek (zoals de gaskinetische theorie) koppelen aan de continu{\"u}m-gebaseerde vloeistofvolume methode om deze lacune in ons model in de toekomst te overbruggen.

In \textbf{hoofdstuk~\ref{chap:TaylorCulick}} vinden wij dat zelfs wanneer het omringende medium interageert met de Taylor-Culick terugtrekking van een film, de film nog steeds met een constante snelheid terugtrekt, op voorwaarde dat de film lang genoeg is om eindige filmgrootte en inwendige viskeuze effecten te vermijden.  Zowel de traagheid als de viscositeit van de omgeving beïnvloeden echter de grootte van deze constante snelheid. Voor de veralgemeende Taylor-Culick terugtrekkingen geldt dat zelfs wanneer de omringende delen een verwaarloosbare viscositeit hebben, zij toch het terugtrekproces beïnvloeden door inerti{\"e}le (toegevoegde massa achtige) effecten. Voor een zeer viskeuze omgeving daarentegen dicteert de viskeuze dissipatie de schaal van de terugtreksnelheid. De precieze aard van deze variatie hangt af van de geometrie van de canonische configuratie in kwestie. Bijvoorbeeld, voor terugtrekkende vellen ondergedompeld in een viskeuze olie, schaalt de terugtreksnelheid met de visco-capillaire snelheid (d.w.z., het capillair getal is een constante). Voor vellen die zich terugtrekken aan een olie-lucht grensvlak vertoont het capillair getal echter een power-law gedrag met de dimensieloze viscositeit (Ohnesorge getal) van het omringende viskeuze medium.

Om de Taylor-Culick terugtrekkingen aan een vloeistof-gas vrij-oppervlak te onderzoeken in hoofdstuk~\ref{chap:TaylorCulick}, ontwikkelen we een precursor film-gebaseerde drie-vloeistof vloeistofvolume methode die het experimenteel waargenomen schalingsgedrag zeer goed weergeeft. In een breder perspectief kan men deze methode gebruiken om verschillende verspreidingsfenomenen op te helderen, zowel op kleine als op grote schaal, zoals respectievelijk druppel-film interacties in het inkjet drukproces en laattijdige verspreiding tijdens olielekkage. Deze numerieke aanname is echter alleen van toepassing wanneer het thermodynamisch gunstig is voor een van de vloeistoffen om zich te verspreiden over de andere vloeistoffen waarmee het in contact komt, d.w.z. dat het een positieve verspreidingscoëfficiënt heeft. Uitbreiding van deze methode tot veralgemeende driefasige contactlijnbewegingen zal naar verwachting interessante resultaten opleveren. Het huidige drie-fasen model kan ook omgaan met verschillende oppervlaktespanningskrachten voor de drie grensvlakken en kan worden gebruikt als basismodel om multi-fysische aspecten te incorporeren, zoals Marangoni stromingen en multicomponent systemen.

In \textbf{hoofdstuk~\ref{chap:BurstingBubbleVP}}, hebben wij aangetoond dat de invloed van viscoplasticiteit op het capillair-gedreven barsten van een luchtbel aan een vloeistof-gas vrije-oppervlak tweevoudig is: (i) het manifesteert zich als een verhoging van de effectieve viscositeit om de capillaire golven te dempen die het barstproces regelen, en (ii) de plasticiteit van het medium verzet zich tegen alle pogingen om het vrije-oppervlak te vervormen. Onmiddellijk na het barsten leiden de grote capillaire spanningen, geloka- liseerd op het snijpunt van de belholte en het vrije-oppervlak, tot een trein van capillaire golven die langs de belholte naar beneden beweegt. In vloeistoffen met lage vloeispanningen volgen deze golven nog steeds hetzelfde gedrag als hun Newtoniaanse tegenhanger. Vervolgens leidt het instorten van de holte tot een Worthington straal die in druppels kan breken ten gevolge van de Rayleigh-Plateau instabiliteit. Voor vloeistoffen met een grote vloeispanning echter verdwijnen de capillaire golven en de Worthington jet. Opbrengst-druk vloeistoffen kunnen vervormingen volhouden. Bijgevolg keert de holte, zelfs na lang wachten, nooit terug naar de nulconfiguratie van de oppervlakte-energie (een vlak vrij-oppervlak). Voor vloeistoffen met hoge vloeispanning kan de plasticiteit van het medium zelfs de capillaire golven overwinnen die het vrije-oppervlak trachten op te leveren, en zo een zoo van uiteindelijke kratervormen bevriezen.
