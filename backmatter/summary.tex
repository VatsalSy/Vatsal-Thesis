\chapter{Summary}

This thesis investigates several free-surface phenomena to illustrate the role of viscous stresses. In part~\ref{PartA} (chapters~\ref{chap:DropForces}--\ref{chap:DropOnDrop}), we study the impact of spherical liquid drops on non-wetting substrates, while in part~\ref{PartB} (chapters~\ref{chap:TaylorCulick}--\ref{chap:BurstingBubbleVP}), we focus on capillary-driven retraction of films and bursting of free-surface bubbles. 

In \textbf{chapter~\ref{chap:DropForces}}, we study water drops impacting non-wetting substrates and find that not only is the inertial shock at impact associated with a distinct peak in the temporal evolution of the normal force, but so is the jump-off, which was hitherto unknown. Furthermore, the inertial pressure force sets the magnitude of both these peaks in the normal reaction force. Surprisingly, some low-velocity impacts can lead to a remarkably high second peak in the normal force, which can even be three times larger than the first. This enhancement can be attributed to the collapse of an air cavity inside the liquid drop leading to singular Worthington jets. Our results thus give a fundamental understanding of the drop impact dynamics on a non-wetting surface and the forces associated with it. Such insight is crucial to developing countermeasures to the failure of superhydrophobicity in technological applications (for example, by avoiding the regime with high impact forces).

In \textbf{chapter~\ref{chap:DropViscousBouncing}}, we delineate the bouncing to the non-bouncing transition of drops falling on a non-wetting substrate. Throughout the drop impact process, viscous dissipation enervates internal momentum. A drop will cease bouncing and stay on the substrate if its upward momentum (driven by capillarity and resisted by viscous stresses) after the drop impact, spreading, and retraction process is insufficient to overcome gravity. Indeed, gravity and viscosity conspire to inhibit drop bouncing off non-wetting substrates. We further observe that close to this transition, the rebound process is independent of the impact parameters. This observation disentangles the later stages of the rebound from the initial impact dynamics. These results are helpful in applications where drop bouncing must be suppressed, for example, inkjet printing, cooling applications, pesticides application, and criminal forensics.

In \textbf{chapter~\ref{chap:DropBouncingOnFilm}}, we investigate drops bouncing off viscous liquid films that mimic atomically smooth substrates. The repellent behavior of such substrates requires the presence of an air layer trapped between the impacting drop and the liquid film. Drops impacting on viscous liquid films show two distinct bouncing regimes: (i) the substrate--independent and (ii) substrate--dependent bouncing. In the former, the impact dynamics are not affected by the presence of the viscous film owing to its high viscosity or negligible thickness. However, in the latter, both the drop and film properties influence the rebound dynamics and govern the bouncing to non-bouncing transition. On the other hand, within the substrate--independent limit, repellency is suppressed once the drop viscosity exceeds a critical value as on superamphiphobic substrates discussed in chapter~\ref{chap:DropViscousBouncing}.

In \textbf{chapter~\ref{chap:DropOnDrop}}, we find that in the presence of a non-wetting substrate, the drop-on-drop impact results in five rebound scenarios, four of which do not involve coalescence. The impacting drop lifts a lazy sessile one in two of the four rebound scenarios. If sufficient energy is transferred between the drops, both drops can take off the substrate, while in some cases, the impacting drop kicks the sessile drop off the substrate but itself cannot bounce. Furthermore, one-to-one comparisons between the experimentally and numerically determined drop boundaries and center of mass mechanical energies illustrate the power of the direct numerical simulations for quantitatively predicting the dynamics of drop-on-drop impact. Insights from chapters~\ref{chap:DropBouncingOnFilm} and~\ref{chap:DropOnDrop} are essential because, for most industrial applications, such as inkjet printing or additive manufacturing, droplets are deposited on a pre-existing layer of another drop or film. Hence, the relative precision of the drop deposition and its shape evolution may decide the success or failure of these devices. 

Note that although we do direct numerical simulations of the several drop impact scenarios in chapters~\ref{chap:DropForces}--\ref{chap:DropOnDrop}, we do so using the continuum equations. Consequently, our numerical method is inadequate to predict whether or not a drop will coalesce with another drop or the substrate. We take this information from the experiments. Indeed the coalescence or non-coalescence of interfaces depends on several multi-physics aspects, including surface asperities and van der Waals forces. Prediction of such coalescence (or rupture) behaviors is beyond the scope of the present thesis. Still, one can couple a consistent molecular dynamics technique (like gas kinetic theory) with the continuum-based volume of fluid method to bridge this lacuna in our model in the future.

In \textbf{chapter~\ref{chap:TaylorCulick}}, we show that even when the surrounding medium interacts with the Taylor-Culick retraction of a film, the film still retracts with a constant velocity, provided that it is long enough to avoid finite film size and internal viscous effects. However, both the inertia and viscosity of the surroundings influence the magnitude of this constant velocity. Even when the surroundings have negligible viscosity, they still influence the retraction process through inertial (added mass-like) effects. On the other hand, for highly viscous surroundings, viscous dissipation dictates the retraction velocity scale. The exact nature of this variation depends on the geometry of the canonical configuration in question. For example, for retracting sheets submerged in a viscous oil, the retraction velocity scales with the visco-capillary velocity (i.e., the capillary number is a constant). However, the capillary number shows a power-law behavior for sheets retracting at an oil-air interface with the dimensionless viscosity (Ohnesorge number) of the surrounding viscous medium. 

To investigate Taylor-Culick retractions at a liquid-gas free-surface in chapter~\ref{chap:TaylorCulick}, we develop a precursor film-based three-fluid volume of fluid method that captures the experimentally-observed scaling behavior very well. In a broader perspective, one can use this method to elucidate several spreading phenomena, both at small and large scales, such as, drop-film interactions in the inkjet printing process and late time spreading during oil spillage, respectively. However, this numerical assumption is applicable only when it is thermodynamically favorable for one of the fluids to spread over the other fluids it comes in contact with, i.e., it has a positive spreading coefficient. Indeed, extending this method to generalized three-phase contact line motions is expected to yield interesting results. The current three-fluid model can also handle different surface tension forces for the three interfaces and can be used as a base model to incorporate multi-physical aspects, such as Marangoni flows and multicomponent systems.

In \textbf{chapter~\ref{chap:BurstingBubbleVP}}, we reveal that the influence of viscoplasticity on the capillary-driven bursting of a bubble at a liquid-gas free-surface is twofold: (i) it manifests as an increase in effective viscosity to attenuate the capillary waves that control the bursting process, and (ii) the plasticity of the medium resists any attempts to deform its free-surface. Immediately after bursting, the large capillary stresses localized at the intersection of the bubble cavity and the free-surface result in a train of capillary waves that travel down the bubble cavity. In liquids with low yield stresses, these waves still follow the same behavior as their Newtonian counterpart. Subsequently, the cavity collapse leads to a Worthington jet that might break into droplets owing to the Rayleigh-Plateau instability. However, the capillary waves and the Worthington jet vanish for liquids with a large yield stress. Furthermore, yield-stress fluids can sustain deformations. Consequently, even after waiting a long time, the cavity never returns to its zero surface energy configuration (a flat free-surface). For high yield-stress liquids, the plasticity of the medium can even overcome the capillary waves that try to yield the free surface, thus freezing a zoo of final crater shapes. 
