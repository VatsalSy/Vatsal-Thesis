\chapter[Lifting a sessile oil drop with an impacting one]{Lifting a sessile oil drop with an impacting one\footnote{Published as: Olinka Ram{\'i}rez-Soto, \textbf{Vatsal Sanjay}, Detlef Lohse, Jonathan T. Pham, and Doris Vollmer. \textit{Lifting a sessile oil drop with an impacting one}, Sci. Adv. \textbf{6}, eaba4330 (2020). Experiments are done by Ram{\'i}rez-Soto; simulations by Sanjay; analysis by both; and supervision by Lohse, Pham, and Vollmer. Writing and proofread by everyone.}}
\label{chap:DropOnDrop}
\chaptermark{Drop-on-drop impact} %this is used at the top of your page to denote the chapter, used in cases where the title is long

Colliding drops are widely encountered in everyday technologies and natural processes, from combustion engines and commercial sprays to raindrops and cloud formation. The outcome of a collision depends on many factors, including the impact velocity and the degree of head-on alignment, in addition to intrinsic properties like surface tension. Yet little is known about the dynamics of an oil drop impacting an identical sessile drop sitting on a low-wetting surface. We experimentally and numerically investigate such a binary impact dynamics of low surface tension oil drops on a superamphiphobic substrate. We observe five rebound scenarios, four of which do not involve coalescence. We describe two previously unexplored cases for sessile oil drop lift-off, resulting from a drop-on-drop impact event. The simulations quantitatively reproduce all rebound scenarios and enable quantification of the velocity profiles, the energy transfers, and the viscous dissipation. Our results illustrate how varying the offset from head-on alignment and the impact velocity results in controllable rebound dynamics for low surface tension drop collisions on superamphiphobic surfaces. 

\clearpage

\section{Introduction}\label{Ch5:Introduction}

When a liquid drop impacts a sessile one of an identical liquid, it is intuitively expected that both drops coalesce. This process is commonly observed in day-to-day examples, such as rain or drops from a leaky faucet. However, coalescence can be obstructed by a thin layer of air between the two drops \citep{Wadhwa2013, kavehpour-2015-arfm, finotello2018dynamics}. Insufficient thinning of this air layer during impact even enables water drops to bounce from perfectly hydrophilic surfaces, which they would otherwise wet \citep{kolinski2014drops, de2015wettability,latka2018drop}. In the late 1800s, \citet{reynolds1881floating} noticed that water drops can glide over a pool because of this air layer. Analogously, a vapor layer also governs the Leidenfrost effect \citep{Boerhaave1732, leidenfrost1756, tran2012drop, quere2013leidenfrost, adera2013non, pham2017spontaneous}, where a drop hovers over a superheated surface. As a result, drop bouncing, coalescence, and spreading can all be observed depending on the intrinsic properties of the liquid, as well as external parameters, such as the background pressure, collision velocity, and the impact parameter describing whether the collision is head-on or off-centered \citep{jiang1992, qian1997, tang2012bouncing, sykes2020surface}. Despite this progress in the experimental characterization of the impact dynamics, a quantitative modelling of the velocity fields and energy transfer is lacking, especially for non-aqueous liquids. 

Drop impact on surfaces, and the outcome of the collision, is of practical importance for many situations. For example, in agriculture, it is essential to ensure that pesticides and other chemicals sprayed on wet leaves do not roll off and contaminate the surroundings \cite{bergeron2000}. Surfactants are often added to lower or tune interfacial tensions. Impact of low interfacial tension drops are encountered in spray coating, inkjet printing, and additive manufacturing of low surface tension liquids \citep{sturgess2017, skylar-scott2019}. On the other hand, removal of drops is desirable for car windows \citep{lv2014bio} and self-cleaning of surfaces. On superhydrophobic surfaces, a water drop impacting another one can lead to drop removal after coalescence if sufficient energy is exchanged between them during impact without viscous dissipation \cite{farhangi2012, damak2018}. Alternatively, even without an impact, the coalescence of droplets can lead to jumping if the excess surface energy released is efficiently transferred to the kinetic energy  \citep{boreyko2009, liu2014, mouterde2017merging, lecointre2019ballistics}.

While several reports exist on understanding how a water drop impacts a sessile water drop on a surface \cite{damak2018, graham2012, farhangi2012, castrejon-pita2013, sykes2020surface}, the dynamics of a low surface tension oil drop impacting an oil drop on a non-wetting surface remains unexplored. A better understanding is, for example, desired in the emerging field of additive manufacturing. For example, in 3D printing, which is one of the widely used additive manufacturing techniques, the relative precision of the drop deposition and its shape evolution may decide the success or failure of a printed device. It has been shown that the collisional dynamics of free-flying oil drops offer more diverse outcomes than those of water drops \citep{jiang1992, qian1997}. Consequently, a number of questions arise. Do these collisions show rich dynamics also in the presence of a low-wetting surface? What are the outcomes of the drop-on-drop impact of oil on a superamphiphobic surface? How is energy transferred between the drops? Intuitively, the rebound of oil drops from a surface by impact with another oil drop seems more difficult than water for the following reasons. (i) The surface tension  of most hydrocarbon oils $\left(\approx 25\,\si{\milli\newton}/\si{\meter}\right)$ is significantly lower than that of water $\left(72\,\si{\milli\newton}/\si{\meter}\right)$ which reduces the transfer of surface energy to kinetic energy during the coalescence. This transfer inefficiency implies that the drops have less energy to rebound. (ii) Sessile oil drops typically have a large contact size. On a flat surface, the receding contact angle is typically below $60\si{\degree}$ and often close to zero \cite{tavana2005}. Consequently, receding oil drops easily rupture before coming off the surface. (iii) On a superamphiphobic surface, oil drops display large apparent contact angles \cite{tuteja2007, pan2018}. However, the true liquid-solid contact angle is still small, leaving oil drops in a metastable state; that is, they can percolate into the surface \cite{papadopoulos2016}. Moreover, pressure as low as a few hundred pascals is sufficient to transition the drop from the metastable Cassie state to a state where the drop wets the surface thoroughly \cite{tuteja2007, butt2013}. The energy threshold needed for this transition is related to the so-called impalement pressure that depends on details of the coating and the liquid under investigation \cite{papadopoulos2016}. (iv) The low surface tension of oil means that the drop is easily deformable, which may give rise to enhanced viscous dissipation and energy loss upon impact. The drop can also locally impale the surface during impact \cite{deng2013, moevius2014}

In this chapter, we experimentally and numerically investigate the dynamics of a low surface tension oil drop impacting a sessile drop of the same liquid, resting on a superamphiphobic surface (figure~\ref{ChDoD:fig1}a). Indeed, we find that the impacting oil drop can lift the resting drop off the surface, without ever coalescing. Notably, we find four rebound scenarios without coalescence: (i) both drops rebound, (ii) two scenarios where the impacting drop rebounds while the sessile drop remains on the substrate, and (iii) the sessile drop rebounds while the impacting drop remains on the surface. We illustrate how these impact outcomes are governed by the Weber number ($\Wen$, ratio of inertial to capillary stresses) and the extent of dimensionless offset from a head-on collision ($\chi = d/(2R)$, where $R$ is the radius of each drop, and $d$ is the distance between their centers of masses, see figure~\ref{ChDoD:fig1}b). Direct numerical simulations provide a quantitative description of (i) the velocity of both drops and of the surrounding vapor phase, (ii) how energy is transferred between the two drops during impact, and (iii) the viscous dissipation during impact and rebound. This allows for a quantitative comparison of experimental and numerical data of the rebound dynamics.

This chapter is organized as follows: \S~\ref{Ch5:Method} briefly describes the experimental and numerical methods. We then explore the drop-on-drop impact by elucidating the experimental observations in \S~\ref{Ch5:Exp}, followed by direct numerical simulations in \S~\ref{Ch5:DNS} where we reproduce the experimentally observed regimes and delve into the process dynamics using internal flow inside the two drops and the energy transfers between them. Lastly, the chapter ends with conclusion and outlook in \S~\ref{Ch5:Conclusion}.

\section{Method}\label{Ch5:Method}
\begin{figure}
	\centering
	\includegraphics[width=\textwidth]{Ch4Figures/Fig1_v4.eps}
	\caption{Experimental approach and the sessile drop: (a) Sketch of the experimental setup for binary drop impact on superamphiphobic surfaces. The needle is fixed to set the impacting height in the $Z$ direction, and the relative distance between the sessile and impacting drops. The sessile drop is first centered along the $YZ$ plane. Then the impacting drop is dispensed from the needle while the impact event is monitored with camera 2.  Camera 1 is used to determine the relative positions of the drops in the $X$ direction. The cameras and the light sources are aligned to observe the impact both in the $XZ$ and $YZ$ planes. Insets: (i) scanning electron micrograph (SEM) image of a soot-templated surface at two magnifications. (ii) hexadecane drop ($\mathcal{V} \approx 3\,\si{\micro\litre}$) resting on the superamphiphobic surface. The orange contour is the solution of the Young-Laplace equation \cite{book-degennes} for a corresponding Bond number $\Bon = 0.3$.  (iii) confocal microscopy image showing a drop of hexadecane on the superamphiphobic surface (the blue line marks its approximate position). The image illustrates the apparent contact angle of the drop with the surface ($\Theta^{\text{app}} \approx 164\si{\degree}$). The image is taken in reflection mode, i.e., no dye was added to the hexadecane. Reflection of light results from the differences between the refractive indices of hexadecane ($1.43$), air ($1.0$), glass and silica ($\approx 1.46$). The superamphiphobic layer consists mostly of air, and thus its refractive index is close to $1$. Therefore, the horizontal glass-superamphiphobic layer and the hexadecane-superamphiphobic layer interfaces are clearly visible. The superamphiphobic layer itself is visible as a diffuse pattern, resulting from the reflection of light from the silica nanoparticles.  (b) Image showing an off-center collision. The impact parameter is $\chi = d/(2R)$. }
	\label{ChDoD:fig1}
\end{figure}

In our experiments, a sessile oil drop is gently positioned on a superamphiphobic surface and then impacted with an identical second oil drop (figure~\ref{ChDoD:fig1}a). The superamphiphobic surface is composed of a $\sim 20\,\si{\micro\meter}$ thick layer of templated candle-soot \cite{pham2017spontaneous, deng2012}. Candle soot consists of a porous network of $50 \pm 20\,\si{\nano\meter}$ sized carbon nanobeads. Making use of chemical vapor deposition (CVD) of tetraethyl orthosilicate (TEOS) catalyzed by ammonia, a $~ 25\,\si{\nano\meter}$ thick layer of silica is deposited over the porous nanostructures to increase the mechanical stability of the fragile network  (figure~\ref{ChDoD:fig1}a). The soot-templated silica network is fluorinated with trichloroperfluoroctylsilane to lower the surface energy, producing a superamphiphobic surface which repels water and most oils. As a model oil, we use hexadecane for its low surface tension, low volatility, homogeneous properties and Newtonian behavior. A drop of hexadecane (figure~\ref{ChDoD:fig1}a-ii) exhibits an apparent contact static angle of $\Theta^{\text{app}} = 164\si{\degree} \pm 1\si{\degree}$. For further details of the drop-substrate contact angle, see appendix~\ref{Ch5:appContactAngles}.

For our drop impact studies, a sessile drop of hexadecane is gently placed on this superamphiphobic surface with a needle connected to a syringe pump (dosing rate: $2\,\si{\milli\liter}/\si{\hour}$). When gravity exceeds the drop-needle adhesion, the drop releases from the needle; this results in a drop volume of $V \approx 3\,\si{\micro\liter}$ (figure~\ref{ChDoD:fig1}a). This volume corresponds to a Bond number of $0.3$ ($\Bon = \rho_lgR^2/\gamma$, ratio of gravitational to capillary stresses, where $\rho_l$ is the density of the liquid, $g$ is the gravitational acceleration, and $R$ is the radius of a spherical droplet of identical volume). Note that a low Bond number implies that a spherical cap can describe the drop. However, it does not provide insight on whether the drop passes the Cassie-to-Wenzel transition. Nonetheless, the shape of the drop is important as it forms the initial condition for the numerical simulation. To calculate and confirm this shape numerically, we solve the Young – Laplace equation \cite{book-degennes}. The shape matches well with the experiments (see the orange contour in figure~\ref{ChDoD:fig1}a-ii). The substrate is then translated laterally to position the sessile drop in the $X$ and $Y$ directions.

At an identical dosing rate, a second drop is released with an identical volume, $\mathcal{V} \approx 3\,\si{\micro\liter}$, which impacts the sessile drop. The control parameters of the drop collision, determining the outcome, are the Weber number ($\Wen$), which is related to the impact velocity $\left(V\right)$, and the impact parameter $\left(\chi\right)$, which describes the offset from head-on alignment of the two colliding drops. The impact velocity $V$ is controlled by varying the height of the dispensing needle from the substrate (figure~\ref{ChDoD:fig1}a). The corresponding Weber number $\Wen = \rho_lV^2R^2/\gamma$ compares fluid inertia and surface tension, where $\rho_l = 770\,\si{\kilogram}/\si{\cubic\meter}$ is the density of the hexadecane and $\gamma = 27.5\,\si{\milli\newton}/\si{\meter}$ is its surface tension coefficient. In our experiments, the Weber number ranges from $0.02$ to $9$. Two synchronized high-speed cameras are perpendicularly positioned to capture the dynamics of the drops in the $X$, $Y$, and $Z$ directions. The impact parameter of the two drops is given by the ratio $\chi = d/\left(2R\right)$, where $d$ is the horizontal offset of the center of masses of the impacting drop and the sessile drop (figure~\ref{ChDoD:fig1}b). Although we cannot exactly predict the impact parameter beforehand, the two camera system allows us to precisely measure the offset from head-on alignment by image analysis. $\chi = 0$ describes a perfect head-on collision whereas $\chi=1$ corresponds to the situation when the two drops merely brush each-other $\left(d = 2R\right)$. 

\section{Experimental observations}\label{Ch5:Exp}

\begin{figure}
	\centering
	\includegraphics[width=\textwidth]{Ch4Figures/Fig2_v3.eps}
	\caption{Snapshots of the impact dynamics: note that the drop labels 1 and 2 are for the impacting and sessile drop, respectively. Six outcomes (Cases I – VI) are observed when varying the impact parameter $\chi$ and the Weber number $\Wen$ independently. The rows correspond to different impact parameter for I-VI. The columns show characteristic stages of the collision process. A: just at collision, B: sessile drop at maximum compression, C: droplet shape just before separation or coalescence. D: final outcome of the impact. Volume of both drops is $3\,\si{\micro\liter}$. Case I, $\Wen = 1.30$ and $\chi = 0.01$: the time stamp for each frame is: $t_{\text{A}} = 0\,\si{\milli\second}$, $t_{\text{B}} = 8\,\si{\milli\second}$, $t_{\text{C}} = 20\,\si{\milli\second}$, $t_{\text{D}} = 25\,\si{\milli\second}$. Case II, $\Wen = 1.53$, $\chi = 0.08$: $t_{\text{A}} = 0\,\si{\milli\second}$, $t_{\text{B}} = 8\,\si{\milli\second}$, $t_{\text{C}} =20\,\si{\milli\second}$, $t_{\text{D}} = 24\,\si{\milli\second}$.  Case III, $\Wen = 1.44$, $\chi = 0.24$: $t_{\text{A}} = 0\,\si{\milli\second}$, $t_{\text{B}} = 8\,\si{\milli\second}$, $t_{\text{C}} = 20\,\si{\milli\second}$, $t_{\text{D}} = 24\,\si{\milli\second}$. Case IV, $\Wen = 1.48$, $\chi = 0.52$: $t_{\text{A}} = 0\,\si{\milli\second}$, $t_{\text{B}} = 5.5\,\si{\milli\second}$, $t_{\text{C}} = 7\,\si{\milli\second}$, $t_{\text{D}} = 21\,\si{\milli\second}$. Case V, $\Wen = 5.84$, $\chi = 0.08$: $t_{\text{A}} = 0\,\si{\milli\second}$, $t_{\text{B}} = 3.75\,\si{\milli\second}$, $t_{\text{C}} = 8.5\,\si{\milli\second}$, $t_{\text{D}} = 25.5\,\si{\milli\second}$. Case VI, $\Wen = 1.43$, $\chi = 0.03$: $t_{\text{A}} = 0\,\si{\milli\second}$, $t_{\text{B}} = 7.5\,\si{\milli\second}$, $t_{\text{C}} = 9\,\si{\milli\second}$, $t_{\text{D}} = 17\,\si{\milli\second}$.}
	\label{ChDoD:fig2}
\end{figure}

\begin{figure}
	\centering
	\includegraphics[width=\textwidth]{Ch4Figures/Fig3_v6.eps}
	\caption{Regime map illustrating the observed cases as a function of the impact parameter $\chi=d/\left(2R\right)$ and Weber number $\Wen$. The top sketches with the respective Roman case number are the possible outcomes after the hexadecane drop impacted on the sessile hexadecane drop. In the image strip, the sessile drop is represented with the solid outline, and the impacting drop with the dotted outline. The arrows represent the direction of motion after impact for each drop. In case III, the impacting drop has a horizontal-curved arrow that represents the rolling of the drop over the sessile drop. In case IV, the impacting drop has two associated arrows. The horizontal-curved arrow represents rolling over the sessile drop and the vertical arrow denotes bouncing after the impact event. Each possible outcome is marked by a color and symbol for identification and corresponds to the sketched cases I-VI. Closed symbols correspond to experiments and open ones to numerical simulations. The transition zones between the different scenario regions are not sharp. The colors assigned to the different cases are meant as a guide to the eye. }
	\label{ChDoD:fig3}
\end{figure}

When varying the offset from head-on alignment $\chi$ and the Weber number $\Wen$ independently, six outcomes for the impact dynamics are observed, termed Cases I-VI (figure~\ref{ChDoD:fig2}). The column A of images is taken just as the collision starts $\left( t = 0\,\si{\milli\second}\right)$ and is used to quantify the impact parameter,$\chi$. Column B is at the point of maximum sessile drop compression, and column C demonstrates the shape of both drop just before they separate or coalesce. Column D illustrates the overall outcome of the collision event. We first consider the outcomes at $We \approx 1.5$ while varying $\chi$. For a near zero $\chi$, Case I is observed, which is a head-on collision (figure~\ref{ChDoD:fig2}-I, supplemental movies~{\color{Myfig}1--3}). During impact, both drops deform and spread radially, and as a result, show axial compression. The kinetic energy of the system is transferred to the surface energies of both the deformed drops. Moving forward in time, both drops start to retract. The sessile drop transfers energy back to the impacting drop in the form of kinetic energy. Upon completion of the collision, the impacting drop bounces off while the sessile drop stays on the substrate. The sessile drop also oscillates, hinting that it retains a part of the energy gained during impact. For a slightly higher offset, $\chi \lesssim 0.15$, Case II is observed (figure~\ref{ChDoD:fig2}-II, supplemental movies~{\color{Myfig}4--6}). The initial collision is similar to Case I in that the drops collide, followed by vertical compression and lateral spreading.  However, unlike Case I, the deformations are no longer symmetric, and the sessile drop also lifts off the surface. The displacement for either drop with respect to the center of mass of the initial sessile drop is in opposing lateral directions. Further increasing of the offset from head-on alignment to $\chi \lesssim 0.5$, the impacting drop glides over the sessile drop and rolls on the substrate, as illustrated by Case III (figure~\ref{ChDoD:fig2}-III, $\chi=0.24$, supplemental movies~{\color{Myfig}7--9}). Unlike Cases I and II, no rebound of the impacting drop is observed. Surprisingly, the sessile instead drop lifts-off the surface. As the impact parameter is increased even further ($\chi > 0.5$, Case IV), the impacting drop still rolls over the sessile drop (figure~\ref{ChDoD:fig2}-IV, supplemental movies~{\color{Myfig}10--12}). However, during retraction, the impacting drop rebounds from the surface while the sessile drop moves along the surface.
	
In the above Cases I-IV, the Weber numbers were kept constant at $We \approx 1.5$ while the offset was varied. However, the outcome of the impact event also varies with the Weber number. To provide a better intuition on how both $\chi$ and $\Wen$ affect the observed outcomes, we plot our data as a phase diagram (figure~\ref{ChDoD:fig3}). For $We \geq 6$, regardless of the impact parameter $\chi$, we find coalescence of the two drops, as illustrated in Case V (figure~\ref{ChDoD:fig2}-V, supplemental movie~{\color{Myfig}13}). In this regime, the air layer between the drops is unstable which results in direct contact and subsequent coalescence. The coalesced drop reaches a maximum spreading diameter during impact (column C in figure~\ref{ChDoD:fig2}-V). During retraction, the drop elongates vertically and ultimately detaches from the surface. Occasionally, drops coalesce without subsequent bouncing (Case VI, figure~\ref{ChDoD:fig2}-VI, supplemental movie~{\color{Myfig}14}). Although this outcome is rarely observed and likely caused by surface defects, we present this result for the sake of completeness to demonstrate all observed outcomes. Moreover, to consider the generality of the scenarios presented for oil-on-oil drop impact, we also tested water-on-water drop impact. Similar scenarios are observed, as illustrated in figure~\ref{Ch5:FigSwater}.

\section{Direct numerical simulations}\label{Ch5:DNS}

\begin{figure}
	\centering
	\includegraphics[width=\textwidth]{Ch4Figures/Fig4_v4.eps}
	\caption{Snapshots from direct numerical simulations: illustration of different phases of drop-on-drop collisions and the subsequent outcomes. (a) Case I: $\left(\chi=0\right)$ impacting drop bounces back and the sessile drops stays on the substrate, (b) Case II: $\left(\chi=0.08\right)$ impacting drop bounces back and the sessile drop lifts-off from the substrate, (c) Case III: $\left(\chi=0.25\right)$ impacting drop stays on the substrate and the sessile drop lifts-off, and (d) Case IV: $\left(\chi=0.625\right)$ impacting drop bounces back and sessile drop stays on the substrate. For all these cases, $\Wen = 1.5$. The drop labels 1 and 2 are for the impacting and sessile drops, respectively. $\tilde{t}$ is the non-dimensionalized time used for the numerical simulations and is given by $\tilde{t} = t/\tau_{\rho\gamma}$ where $\tau_{\rho\gamma} = \sqrt{\left(\rho R^3\right)/\gamma}$ is the inertio-capillary time scale. The absolute values of the normalized velocities vary between zero (white) and twice the inertio-capillary velocity, $V_{\rho\gamma}=\sqrt{\gamma/\left(\rho R\right)}$ (dark blue).}
	\label{ChDoD:fig4}
\end{figure}

\begin{figure}
	\centering
	\includegraphics[width=\textwidth]{Ch4Figures/Fig5_v3.eps}
	\caption{Energy budget: the temporal variation of energy transfer elucidates different stages of the drop-on-drop impact process at $\Wen = 1.5$. Initially, all the energy is stored as the mechanical energy of the impacting drop and surface energy of the sessile drop ($E_0$). Then, the mechanical energy of the system decreases, and is transferred into the surface energy of the drops. This transfer is followed by a recovery stage where surface energy is transferred back into the mechanical energy of the system. A part of the energy is lost as viscous dissipation. This viscous dissipation takes into account the combined energy dissipated both in the liquid drops and the surrounding air. This calculation includes the air-layers between the drops, and between the drops and the superamphiphobic substrate. (a) Case I: $\chi = 0$, (b) Case II: $\chi = 0.08$, (c) Case III: $\chi = 0.25$, and (d) Case IV: $\chi = 0.625$. $E_m$ is the total mechanical energy of the system $\left(E_m = E_k + E_g\right)$, $E_\gamma$ the surface energy of the two drops, and $E_\eta$ the viscous dissipation in the system. Note that the total mechanical energy $\left(E_m\right)$ includes the energy of center of mass of the drops $\left(E_m^{\text{CM}} = E_k^{\text{CM}} + E_g^{\text{CM}}\right)$, where $E_g^{\text{CM}}$ is the gravitational potential energy as well as the oscillation and rotational energies obtained in the reference frame that is translating with the center of mass of the individual drops.}
	\label{ChDoD:fig5}
\end{figure}

Although the experimental observations consistently illustrate how $\Wen$ and $\chi$ dictate the observed impact outcomes, they lack detailed information on the velocity fields and how energy is transferred between the two drops. To access this information, we ran direct numerical simulations (DNS) and compared these results with our experimental data. For simulating non-coalescing droplets, we employ geometric Volume of Fluid (VoF) \citep{popinet2009accurate, basiliskpopinet1} method with two distinct VoF tracers (see \S~\ref{ChDoD:sec:simulationMethodology} for detailed discussions and implementation). This formulation ensures that drops cannot coalesce, reflecting the experimental situation where a finite air layer between the drops is preserved throughout the process.

We first ran four simulations choosing $\Wen$ and $\chi$ values within the regimes for Cases I-IV, as denoted by the open symbols in figure~\ref{ChDoD:fig3}. The results are displayed in figure~\ref{ChDoD:fig4}. The normalized times ($\tilde{t} = t/\tau_{\rho\gamma}$, where $\tau_{\rho\gamma} = \sqrt{\left(\rho_l R^3\right)/\gamma}$ is the inertio-capillary time scale.) correspond to the stages of the process, as described by columns A – D in figure~\ref{ChDoD:fig2}. As is evident from the top rows (orange drops), the simulations reproduce the general collision outcomes consistent with the snapshots of the impact dynamics (figure~\ref{ChDoD:fig3}). Moreover, the direct numerical simulations allow for quantifying the velocity vector fields for each of the cases (figure~\ref{ChDoD:fig4}, bottom rows). These vector fields, combined with a calculation of the energy budget, renders it possible to quantitatively explore the dynamics of the oil drop-on-drop collision process. To account for the kinetic energy $E_k$, gravitational potential energy $E_g$, surface energy $E_\gamma$, and viscous dissipation $E_\eta$, we numerically calculated the total energy of the system as 

\begin{align}
	\label{ChDoD:eq:EnergyBudget}
	E = E_m + E_\gamma + E_\eta,
\end{align}

\noindent where the energies are calculated using a method similar to the one developed by \citet{wildeman2016spreading}.  Note that $E_k$ includes the kinetic energy of the center of mass as well as the oscillation and rotational energies obtained in the reference frame that is translating with the center of mass of the individual drops. The details of these calculations are provided in \S~\ref{ChDoD:sec:EnergyBudgetCalculations}.

While keeping the Weber number at $\Wen = 1$, the cases appear in order from I to IV with increasing offset position from head-on alignment $\chi$. For all cases, the energy is initially contained in the mechanical energy of the impacting drop (i.e., its kinetic and potential energy) and the surface energy of the sessile drop. To describe the system energy of the DNS results presented in figure~\ref{ChDoD:fig4}, we plot the complete energy budget for each case in figure~\ref{ChDoD:fig5}. For convenience of comparison, the energies in figure~\ref{ChDoD:fig5} are normalized with the initial energy of the system.

Let us consider first a head-on collision where $\chi = 0$ (figure~\ref{ChDoD:fig4}a, figure~\ref{ChDoD:fig5}a, and supplementary videos~{\color{Myfig}2--3}, Case I), which is defined by a symmetric configuration. First, the momentum is transferred from the impacting drop to the sessile drop, as the sessile drop deforms. This transfer results in deceleration of the impacting drop. Moreover, the kinetic energy of the impacting drop transforms into the surface energy of the system. This transfer continues until $\tilde{t} = 1.84$ (Fig. 4a: Column B) when the deformation in the two drops is maximum. Even at the moment of maximal deformation of both drops, the kinetic energy remains finite because of rotational flow within the drops (figure~\ref{ChDoD:fig4}a: column B, velocity field) \cite{wildeman2016spreading}. The mechanical energy passes a minimum ($\tilde{t} = 1.84$) when the surface energy is maximal. For $\tilde{t} > 1.84$, the surface energy of the two drops is converted back into kinetic energy. Retraction of the sessile drop is hindered by the impacting one (figure~\ref{ChDoD:fig4}a: column C), directly sitting on top of it. As a result, the sessile drop cannot lift-off from the substrate, but it releases any extra energy by oscillations (supplementary videos~{\color{Myfig}1--2}). During impact, the drops lose approximately 20\% of their initial energy through viscous dissipation (figure~\ref{ChDoD:fig5}a). This dissipation occurs mainly during the initial stages of the process ($\tilde{t} < 3$). It should be noted that the surface tension coefficient ($\gamma$), viscosity ($\eta$), and impact velocity ($V$) all affect viscous dissipation (see chapter~\ref{chap:DropViscousBouncing}). These properties are related to the Ohnesorge number ($\Ohn = \eta/\sqrt{\rho_l\gamma R} \approx 0.03$), which is the ratio of the inertio-capillary to inertio-viscous timescales, and the Weber number, $\Wen = \rho_lV^2R/\gamma \sim \mathcal{O}\left(1\right)$ (see equation~\eqref{ChDoD:eq:VisDissFnct} and \citep{landau2013course, molavcek2012quasi}). The dissipation observed in our case is lower than that reported previously for a single drop impact at comparable $\Ohn$ and $\Wen$ on non-wetting rigid substrates (see chapter~\ref{chap:DropViscousBouncing}). In the case of a single drop impact, the velocity of the drop goes to zero quickly as it approaches a rigid substrate \cite{clanet2004}, leading to high dissipation close to the substrate. In the case of drop-on-drop impact, the sessile drop is deformable, decreasing the deceleration experienced by the impacting drop. As a result, the system retains almost 80\% of its initial energy in the form of mechanical and surface energy of the drops. 

For slightly off-center collisions, where $\chi=0.08$ (figure~\ref{ChDoD:fig4}b, figure~\ref{ChDoD:fig5}b, and supplementary videos~{\color{Myfig}5--6}, Case II), the initial collision is similar to Case I: the drops collide, followed by vertical compression and lateral spreading. However, unlike Case I, the impacting and the sessile drops lift-off from the substrate. This feature results from the loss of axial symmetry of the velocity field for $\chi > 0$. During retraction, transfer of momentum from the compressed sessile drop back to the impacting drop occurs mainly along a vector pointing normal to the apparent contact zone. Moreover, the sessile drop attempts to regain its spherical shape (minimum surface energy state). As a result, the velocity field of the sessile drop is almost parallel to the contact zone, i.e., pointing to the upper left. These opposing orientations of the velocity fields cause the impacting drop to bounce off the sessile drop, and the sessile drop to lift-off from the substrate (see the velocity vector fields in figure~\ref{ChDoD:fig4}b and supplementary video~{\color{Myfig}6}). Viscous dissipation increases as compared to a head-on-collision, but still is maximum during the initial stages of the process owing to the dissipation in the viscous boundary layer as the impacting drop slides over the sessile one ($\tilde{t} < 3.5$, figure~\ref{ChDoD:fig5}b).

As the offset is further increased to $\chi = 0.25$ (figure~\ref{ChDoD:fig4}c, figure~\ref{ChDoD:fig5}c, and supplementary videos~{\color{Myfig}8--9}, Case III), the impacting drop glides over the sessile drop (facilitated by the thin air layer), and sufficient energy is transferred to lift the sessile drop from the substrate. This can be understood from the interplay of the velocity field and the contact time (figure~\ref{ChDoD:fig4}c and supplementary video~{\color{Myfig}9}). The relatively large offset from head-on alignment causes the averaged velocity field of the restoring impacting drop to point downwards, while the velocity field of the sessile drop is pointing upwards. The large deformations of both drops are reflected in the evolution of the surface energy (figure~\ref{ChDoD:fig5}c). These large deformations also cause an increase in the viscous dissipation ($E_\eta$): at the end of the process, almost 50\% of the initial energy is lost.

Finally, if the offset from head-on alignment is increased even more to $\chi = 0.625$ (figure~\ref{ChDoD:fig4}d, figure~\ref{ChDoD:fig5}d, and supplementary videos~{\color{Myfig}11--12}, Case IV), the time of contact is insufficient to transfer enough energy to the sessile drop for lift-off \cite{andrew2017}. Moreover, the vector normal to the drop-drop contact area is farthest from vertical as compared to the normal vectors in other cases. That is, it points nearly horizontal. As a result, the sessile drop rolls along the substrate and the impacting drop instead rebounds from the surface, resembling typical drop-surface impact. In this case, most of the energy is retained by the impacting drop, as illustrated in Fig. 5d. Similar to Case III, viscous dissipation accounts for almost 50\% of the initial total energy. Although in Case I and IV the impacting drop rebounds while the sessile drop remains on the surface, we discriminate between both cases. For Case I, the vector fields are symmetric around the $X = Y = 0$ axis, whereas for Case IV the vector fields are highly asymmetric and the sessile drop rolls along the surface. Furthermore, in Case IV, the impacting drop bounces-off the substrate, as opposed to the sessile drop in Case I. 

\begin{figure}
	\centering
	\includegraphics[width=\textwidth]{Ch4Figures/Fig6_v2.eps}
	\caption{Validation of the numerical code: (a) Case II: both sessile and impacting drop lift-off $\left(\Wen = 1, \chi \approx 0.08\right)$ for t = (i) $0\,\si{\milli\second}$, (ii) $8\,\si{\milli\second}$, (iii) $20\,\si{\milli\second}$, (iv) $24\,\si{\milli\second}$, and (b) Case III: sessile drop lifts-off and impacting drop rolls on the substrate $\left(\Wen = 1, \chi \approx 0.25\right)$ for t = (i) $0\,\si{\milli\second}$, (ii) $8\,\si{\milli\second}$, (iii) $20\,\si{\milli\second}$, (iv) $24\,\si{\milli\second}$. In the subfigures (i) to (iv), overlay of experimental images and DNS results (orange contour) are shown. (v) The mechanical energy of the center of mass $\left(E_m^{\text{CM}}\right)$ calculated from experiments and simulations match within the experimental error. Note that in experiments, we could only keep track of the motion of the center of mass whereas in numerical simulations, the entire velocity field is known. Using this information, we can calculate the overall energy budgets. Here, the total mechanical energy of the drops $\left(E_m\right)$ is shown in solid lines for reference. Error estimated in the experimental data is approximately 20\% of the total energy.}
	\label{ChDoD:fig6}
\end{figure}

These results indicate that the DNS provide a quantitative description of the impact dynamics. At this point, we investigate whether there is a one-to-one match of the experimental data and numerical simulations; this is done by comparing the drop boundaries and experimentally-determined mechanical energies with the numerical predictions. Since we cannot exactly predict the impact parameter experimentally beforehand, we choose the control parameters for the numerical simulations by first analyzing the experimental data. Notably, we achieve a nearly quantitative agreement of the drop boundaries and experimental mechanical energies (figure~\ref{ChDoD:fig6}).The different snapshots in figure~\ref{ChDoD:fig6}(i-iv) refer to the following time steps: (i) at the instant of collision, (ii) sessile drop at maximum compression, (iii) droplet shape just before separation, and (iv) final outcome of the impact. We expect that slight deviations between the experimental and numerically determined drop boundaries result from marginal inaccuracies in the experimental determination of the off-set parameter. However, the agreement is remarkably good, keeping in mind that there are no fitting parameters. 

In figure~\ref{ChDoD:fig6}(a-v) and~\ref{ChDoD:fig6}(b-v), we compare the measured experimental mechanical energies (data points) with those calculated using simulations (dotted lines). The calculated mechanical energies exceed the experimentally determined energies. To understand the origin of this discrepancy, one needs to consider that experimentally, we are only able to measure the vertical and horizontal displacements to approximate the mechanical energy of each drop. The images analysis did not offer an easy route to quantify the contribution of the rotational and oscillation energies that are included in the numerically calculated mechanical energy, $E_m$. Therefore, to test whether neglecting the rotational and oscillation energies in our experiments causes the discrepancy, we calculated the center of mass mechanical energies ($E_m^\text{CM}$) for the two drops numerically (figure~\ref{ChDoD:fig6}a-v and~\ref{ChDoD:fig6}b-v, see \S~\ref{ChDoD:sec:EnergyBudgetCalculations} for details of calculation). The zero of the potential energy $\left(E_g^\text{CM}=0\right)$ refers to the center of mass of the sessile drop at $t = 0$. This implies that $E_g^\text{CM}$ of the sessile drop becomes negative during compression. The center of mass kinetic energy $\left(E_k^\text{CM}\right)$ is added to this value to get $E_m^\text{CM}$, i.e., $E_m^\text{CM} = E_k^\text{CM} + E_g^\text{CM}$. As illustrated in figure~\ref{ChDoD:fig6}(a-v) and~\ref{ChDoD:fig6}(b-v), the numerical results (solid lines) now nearly overlay the experimental results (data points). This holds for both the temporal development of the energy for the sessile drop as well as the impacting drop. Supposedly, the small discrepancies may arise from finite adhesion of the sessile drop to the substrate in the experiments (which is not accounted for in the simulations). An additional source of error may arise from the selection of time $t = 0$. We choose $t = 0$ based on the time instant when the sessile drop starts to feel the presence of the velocity field of the impacting drop, i.e., when the kinetic energy of the center of mass of the sessile drop becomes non-zero. Nevertheless, the remarkable agreement between the experimental and numerical results for the center of mass mechanical energies illustrate that the DNS are able to describe the oil drop-on-drop impact physics accurately. This allows for quantifying the contribution of the rotational and oscillatory energies. As future work, one can also estimate these contributions from experimentally obtained boundaries of the drops by employing the method described in \citet{molavcek2012quasi}.

\section{Conclusions and outlook}\label{Ch5:Conclusion}
By combining systematic experiments with numerical simulations, we illustrate how to predict and control the outcome of binary oil drop impacts on low adhesion surfaces. Four non-coalescing outcomes are attainable by varying the Weber number $\Wen$ and the offset from head-on alignment of the impacting drops $\chi$. One-to-one comparisons between the experimentally and numerically determined drop boundaries and center of mass mechanical energies illustrate the power of the direct numerical simulations for quantitatively predicting the dynamics of drop-on-drop impact. More specifically, our numerical simulations illustrate that these general outcomes are governed by the average direction of the flow velocity vectors during the retraction phase, which are associated with $\Wen$ and $\chi$. In addition, our results indicate that the ability to remove a sessile oil drop from the surface, as in Cases II and III, first requires sufficient energy transfer from the impacting drop and subsequently requires contrasting velocity vector directions of the two retracting drops. Interestingly, our results illustrate that different outcomes exist even when the total dissipative losses of the system are similar, i.e., the alignment of impact alone can be used to determine the recovered energy distribution between the two drops after impact.

%\afterpage{\clearpage}
\section*{Acknowledgments}
We would like to thank Hans-J{\"u}rgen Butt for fruitful discussions, Andrea Prosperetti for insightful discussions on numerical simulations at different stages of the project, Pierre Chantelot for discussion on energy dissipation in single drop impact phenomenon, and Uddalok Sen for proof-reading this chapter. We would also like to thank Michael Kappl for help with the estimation of forces on the substrate, and Abhishek Khadiya for support with initial measurements. 

\begin{subappendices}
	%\subsection{Soot-templated superamphiphobic glass slides preparation }
	%\begin{figure}
	%	\centering
	%	\includegraphics[width=\textwidth]{Ch4Figures/S1.eps}
	%	\caption{Preparation of soot-templated glass superamphiphobic surfaces: Soot particles are deposited on a glass slide. The particles are coated with silica, by applying a chemical treatment with Tetraethoxysilane. To make the particle layer transparent, combustion is induced. With plasma treatment, OH groups are formed to chemically bind the trichloro (1H, 1H, 2H, 2H-perfluorooctyl) silane. }
	%	\label{ChDoD:figS1}
	%\end{figure}
	%
	%The soot-templated superamphiphobic glass slides were made following the process reported previously \cite[see figure~\ref{ChDoD:figS1} and][]{paven2013, paven2014}. The glass slides were sonicated for cleaning with ethanol, acetone, and toluene, for $5\,\si{\minute}$ in each solvent. The glass slides were dried in an oven at $60\si{\degreeCelsius}$. For coating the glass slides with candle soot, the glass slides were hold above the center of the candle flame for approximately $1\,\si{\minute}$. To form a uniform layer of soot particles, the glass slides were oscillatory moved in the horizontal plane. The coated glass slides were stored on a desiccator for $24\,\si{\hour}$. with an open snap cap vial containing $3\,\si{\milli\liter}$ of ammonium and a second vial with $3\,\si{\milli\liter}$ of tetraethoxysilane. Afterwards, the samples were heated for $5\,\si{\hour}$ at $550\,\si{\degreeCelsius}$ in an oven to get transparent substrates. The samples were coated with an approximately $25\,\si{\nano\meter}$ thick silica shell. After activation in an oxygen plasma for $10\,\si{\minute}$, the samples were fluorinated with trichloro (1H, 1H, 2H, 2Hperfluorooctyl) silane on a desiccator for $2\,\si{\hour}$. 
	\section{Contact angle measurements}\label{Ch5:appContactAngles}
	
	\begin{figure}
		\centering
		\includegraphics[width=\textwidth]{Ch4Figures/S2_v5.eps}
		\caption{Sessile oil drop on a superamphiphobic substrate: (a) shadowgraph  and (b-d) confocal images of a hexadecane drop on soot-templated glass slide. The shadowgraph image shows the typical shape of a sessile hexadecane drop during the experiments. The corresponding volumetric radius is $0.9\,\si{\milli\meter}$. The confocal microscopy image illustrates the apparent contact angle of the drop with the surface. The image was taken in reflection mode to allow measuring the contact angle with highest possible accuracy. The measured roll-off angle of a drop of $5\,\si{\micro\liter}$ is $3.2\si{\degree}$ (measured with a goniometer). The apparent contact angle $\Theta^{\text{app}} \approx 164\si{\degree}$, the receding angle $\Theta_r^{\text{app}} \approx 158\si{\degree}$, and the advancing angle $\Theta_a^{\text{app}} \approx 173\si{\degree}$, were measured with a drop of $10\,\si{\micro\liter}$ volume. Ideally, $\Theta_a^{\text{app}}$ should be $180\si{\degree}$. The difference could be attributed to the limited optical contrast.}
		\label{ChDoD:figS2}
	\end{figure}

	A drop of hexadecane (figure~\ref{ChDoD:fig1}a-ii) exhibits an apparent contact static angle of $\Theta^{\text{app}} = 164\si{\degree} \pm 1\si{\degree}$, an apparent receding contact angle of $\Theta_r^{\text{app}} = 158\si{\degree} \pm 3\si{\degree}$, and an apparent advancing contact angle $\Theta_a^{\text{app}} \approx 180\si{\degree}$ \citep{schellenberger2016}, as determined by confocal microscopy (figure~\ref{ChDoD:fig1}a-iii and~\ref{ChDoD:figS2}). Low lateral adhesion of hexadecane is confirmed by measuring a low roll-off angle of $3\si{\degree} \pm 2\si{\degree}$ \cite{quere2002fakir, elsherbini2006}. Roll-off angles measurements were performed using a goniometer OCA 35 for hexadecane drops of $5\,\si{\micro\liter}$. The apparent contact angle was measured with a Leica TCS SP8 confocal microscope, equipped with an HCX PL APO 40x/0.85 dry objective, for a hexadecane drop of $10\,\si{\micro\liter}$. The advancing and receding angles were measured while moving the hexadecane drop with a needle. The needle was supported on a micrometer stage next to the confocal microscope. All angles were measured at least three times and the results are shown in figure~\ref{ChDoD:figS2}.
	
	\section{Water-on-water drop impact}
	For comparison, we also performed head-on and off-center collisions for water drops of a similar diameter than for oil drops. To test the generality of our experiments, we used superamphiphobic textiles instead of soot templated surfaces. Superamphiphobic textiles show slightly worse wetting properties than soot template surfaces. This is reflected in higher roll-off angles. The roll-off angle of water with the surface varied between $13\si{\degree} - 20\si{\degree} $ for the textile surfaces while the roll-off angles varied between $2\si{\degree} - 5\si{\degree} $ on the soot-templated glass. On the other hand, the Cassie-to-Wenzel transition is less likely for water drops than for oil drops. When varying the Weber number $\Wen$ and the impact parameter $\chi$, the same six outcomes could be observed, same as the cases with hexadecane drops (figure~\ref{Ch5:FigSwater})
	
	\begin{figure}
		\centering
		\includegraphics[width=\textwidth]{Ch4Figures/FigureSupplementalWaterImpacts_v2.eps}
		\caption{Snapshots of the impact dynamics of water drops: note that the drop labels 1 and 2 are for the impacting and sessile drop, respectively. Six outcomes (Cases I – VI) are observed when varying the impact parameter $\chi$ and the Weber number $\Wen$ independently. The rows correspond to different impact parameter for I-VI. The columns show characteristic stages of the collision process. A: just at collision, B: sessile drop at maximum compression, C: droplet shape just before separation or coalescence. D: final outcome of the impact. Volume of both drops is $8\,\si{\micro\liter}$. Case I, $\Wen = 0.58$ and $\chi = 0.03$: the time stamp for each frame is: $t_{\text{A}} = 0\,\si{\milli\second}$, $t_{\text{B}} = 10.5\,\si{\milli\second}$, $t_{\text{C}} = 23\,\si{\milli\second}$, $t_{\text{D}} = 36\,\si{\milli\second}$. Case II, $\Wen = 1.23$, $\chi = 0.06$: $t_{\text{A}} = 0\,\si{\milli\second}$, $t_{\text{B}} = 10.5\,\si{\milli\second}$, $t_{\text{C}} =19\,\si{\milli\second}$, $t_{\text{D}} = 30.5\,\si{\milli\second}$.  Case III, $\Wen = 1.14$, $\chi = 0.16$: $t_{\text{A}} = 0\,\si{\milli\second}$, $t_{\text{B}} = 9\,\si{\milli\second}$, $t_{\text{C}} = 15\,\si{\milli\second}$, $t_{\text{D}} = 30\,\si{\milli\second}$. Case IV, $\Wen = 1.47$, $\chi = 0.59$: $t_{\text{A}} = 0\,\si{\milli\second}$, $t_{\text{B}} = 5.5\,\si{\milli\second}$, $t_{\text{C}} = 9\,\si{\milli\second}$, $t_{\text{D}} = 24\,\si{\milli\second}$. Case V, $\Wen = 0.53$, $\chi = 0.002$: $t_{\text{A}} = 0\,\si{\milli\second}$, $t_{\text{B}} = 4\,\si{\milli\second}$, $t_{\text{C}} = 14.5\,\si{\milli\second}$, $t_{\text{D}} = 36\,\si{\milli\second}$. Case VI, $\Wen = 0.68$, $\chi = 0.05$: $t_{\text{A}} = 0\,\si{\milli\second}$, $t_{\text{B}} = 10\,\si{\milli\second}$, $t_{\text{C}} = 12.5\,\si{\milli\second}$, $t_{\text{D}} = 21\,\si{\milli\second}$.}
		\label{Ch5:FigSwater}
	\end{figure}

	\section{Simulation methodology}
	\label{ChDoD:sec:simulationMethodology}
	
	We use a finite volume method based partial differential equation solver, Basilisk C \citep{basiliskpopinet1} for numerical simulation of incompressible Navier-Stokes equations,
	
	\begin{align}
		\label{ChDoD:eq:continuity}
		\boldsymbol{\nabla\cdot v} &= 0,\\
		\label{ChDoD:eq:NS}
		\frac{\partial \boldsymbol{\tilde{v}}}{\partial\tilde{t}} + \left(\boldsymbol{\tilde{v}\cdot\tilde{\nabla}}\right)\boldsymbol{\tilde{v}} &= \frac{1}{\tilde{\rho}}\left(-\boldsymbol{\tilde{\nabla}}\tilde{p} + \Ohn\boldsymbol{\tilde{\nabla}\cdot}\left(2\tilde{\eta}\boldsymbol{\tilde{\mathcal{D}}}\right) + \tilde{\kappa}\tilde{\delta}_s\boldsymbol{\hat{n}}\right) - \Bon\boldsymbol{\hat{Z}},
	\end{align}
	
	\noindent where the velocity $\boldsymbol{v}$ and pressure $p$ fields are non-dimensionalized using the inertio-capillary velocity $\left(V_{\rho\gamma} = \sqrt{\gamma/\left(\rho_lR\right)}\right)$ and capillary pressure $\left(P_{\gamma} = \gamma/R\right)$, respectively. All length scales are normalized using the radius of the impacting drop ($R$). In equation~\eqref{ChDoD:eq:NS}, $\boldsymbol{\mathcal{D}}$ is the deformation tensor (i.e., the symmetric part of the velocity gradient tensor, $\boldsymbol{\tilde{\nabla}\tilde{v}}$) and $\tilde{\kappa}\tilde{\delta}_s\boldsymbol{\hat{n}}$ represent the singular ($\delta_s = 1$ at the interfaces and $0$ otherwise) surface tension force, where $\kappa$ and $\boldsymbol{\hat{n}}$ are the interfacial curvature and normal, respectively. Since we do not vary the type of liquid during and the volume of drops in our experiments or simulations, Ohnesorge number ($\Ohn = \eta_l/\sqrt{\rho_l\gamma R} = 0.0216$) and Bond number ($\Bon = \rho_l g R^2/\gamma = 0.308$) remain constant. Lastly, in the simulations, the impact velocity is characterized by the impact Weber number ($V = \sqrt{We}$).
	
	We use the geometric volume of fluid (VoF) \citep{popinet2009accurate, basiliskpopinet1} method for interface tracking. Consequently, one-fluid approximation \cite{prosperetti2009computational, tryggvason2011direct} is used in the solution of the Navier-Stokes momentum equation (equation~\eqref{ChDoD:eq:NS}). 
	
	To impose the condition of non-coalescence of the drops, same as chapter~\ref{chap:DropBouncingOnFilm}, different VoF tracers are used for the two droplets (equation~\ref{ChDoD:eq:vof1}). The use of two different tracers, along with interface reconstruction, ensures that there is always a thin air layer (thickness $\sim \Delta_1$, where $\Delta_1=R/256$ is the size of smallest grid cell in the simulation domain). Additionally, in order to model the superamphiphobic substrate, it is assumed that there is a thin air layer (thickness $\sim \Delta_2$, where $\Delta_2 = R/512$ is the smallest grid cell near the substrate) between the drops and the substrate. All other boundaries are assumed to have no flow and free slip condition. We ensure convergence by comparing the viscous dissipation of the system and have chosen $\Delta$ such that the difference between consecutive simulations is small. The properties, such as density and viscosity are calculated using the VoF arithmetic property equations (equation~\ref{ChDoD:eq:vof2}, where $A_{gl}$ is the ratio of properties of gas and liquid).
	
	\begin{align}
		\label{ChDoD:eq:vof1}
		\left(\frac{\partial}{\partial t} + \boldsymbol{v\cdot}\nabla\right)\{\Psi_1,\Psi_2\} = 0&,\\
		\label{ChDoD:eq:vof2}
		\tilde{A} = A_{gl} + \left(1 - A_{gl}\right)\left(\Psi_1 + \Psi_2\right)& \quad\forall\,A\in\left[\rho, \eta\right],
	\end{align}
	
	\section{Energy budget calculations}\label{ChDoD:sec:EnergyBudgetCalculations}
	
	In this section, we discuss the different equations that we have used to calculate different energies non-dimensionalized by the capillary energy scale $\tilde{E}_i = E_i/\left(\gamma R^2\right)$. First, we discuss the calculation of energies of the center of mass of the drops $\left(\tilde{E}_m^\text{CM}\right)$,
	
	\begin{align}
		\label{ChDoD:eq:energyCM}
		\tilde{E}_m^\text{CM} = \tilde{E}_k^\text{CM} + \tilde{E}_g^\text{CM},
	\end{align}
	
	\noindent where $\tilde{E}_k^\text{CM}$ and $\tilde{E}_g^\text{CM}$ are the center of mass kinetic energy and potential energy respectively. For these calculations, we first need to find the magnitude of velocity and position of the center of mass for each drop, 
	
	\begin{align}
		\boldsymbol{\tilde{v}^\text{CM}} = \frac{\int_{\tilde{\Omega}}{\boldsymbol{\tilde{v}}d\tilde{\Omega}}}{\tilde{\Omega}},\\
		\tilde{Z}^\text{CM}= \frac{\int_{\tilde{\Omega}}{\tilde{Z}d\tilde{\Omega}}}{\tilde{\Omega}}
	\end{align}
	
	\noindent where, $d\Omega$ is the differential fluid volume. Further, $E_k^\text{CM}$ and $E_g^\text{CM}$ can be calculated,
	
	\begin{align}
		&\tilde{E}_k^\text{CM}=\frac{2}{3}\pi\left(\boldsymbol{\tilde{v}^\text{CM}\cdot \tilde{v}^\text{CM}}\right),\\
		&\tilde{E}_g^\text{CM} = \Bon \tilde{Z}^\text{CM}.
	\end{align}
	
	The overall energy budget consists of the total mechanical energy $\tilde{E}_m = \tilde{E}_k + \tilde{E}_g$, the surface energy $\tilde{E}_\gamma$, and the energy dissipation $\tilde{E}_\eta$, calculated as follows:
	
	\begin{align}
		\label{ChDoD:eq:TotalKE}
		&\tilde{E}_k = \int_{\tilde{\Omega}}\frac{1}{2}\tilde{\rho}\left(\boldsymbol{\tilde{v}\cdot\tilde{v}}\right)d\tilde{\Omega},\\
		\label{ChDoD:eq:TotalGE}
		&\tilde{E}_g = \Bon\int_{\tilde{\Omega}}\tilde{\rho}\tilde{Z}d\tilde{\Omega},\\
		\label{ChDoD:eq:TotalSE}
		&\tilde{E}_\gamma = \int_{\tilde{\mathcal{A}}}d\tilde{\mathcal{A}},\\
		\label{ChDoD:eq:TotalDE}
		&\tilde{E}_\eta = \int_0^{\tilde{t}}\tilde{\xi}_\eta d\tilde{t}
	\end{align}
	
	In equations~\eqref{ChDoD:eq:TotalKE} and~\eqref{ChDoD:eq:TotalGE}, energies of both the drops as well as the surrounding air medium are considered. Noticing that the density ratio of air to liquid, $\rho_{gl} = 1/770 \ll 1$ and that the domain is fixed in volume, the change in gravitational potential energy of the air medium is negligible. This implies that $E_g = E_g^\text{CM}$. The contributions from the surrounding air to $E_k$ and $E_\eta$ are also very small but we include them here for completeness. In equation~\ref{ChDoD:eq:TotalSE}, $d\mathcal{A}$ represents a differential surface. Lastly, equation~\ref{ChDoD:eq:TotalDE} gives the total viscous dissipation in the system. In this equation, $\xi_\eta$ denotes the viscous dissipation function given by
	
	\begin{align}
		\label{ChDoD:eq:VisDissFnct}
		\tilde{E}_\eta = \Ohn\int_{\tilde{\Omega}}2\tilde{\eta}\left(\boldsymbol{\tilde{\mathcal{D}}:\tilde{\mathcal{D}}}\right)d\tilde{\Omega}.
	\end{align}
	
	\section{Code availability}
	
	All codes used in this chapter are permanently available at \citet{basiliskvatsalDoD}.
	
	\section{Supplemental movies}
	These supplemental movies are available at \citet[\href{https://youtube.com/playlist?list=PLf5C5HCrvhLEX9_VlqlK99mdtxtu1l-HQ}{external YouTube link,}][]{vatsalDoDsuppl}. 
	
	\begin{enumerate}
		\item[SM1:] $\left(\Wen \approx 1.30\,\&\,\chi \approx 0.01\right)$ Experimental video of Case I for hexadecane drops: bouncing of impacting drop.
		\item[SM2:] $\left(\Wen = 1.50\,\&\,\chi = 0\right)$ Simulation video of Case I for hexadecane drops: bouncing of impacting drop.
		\item[SM3:] $\left(\Wen = 1.50\,\&\,\chi = 0\right)$ Simulation video showing velocity vectors of Case I for hexadecane drops: bouncing of impacting drop. The two-dimensional contour represents the slice $Y = 0$. Time is normalized by the capillary time scale, $\tau_{\rho\gamma} = \sqrt{\left(\rho_l R_0^3\right)/\gamma}$.
		
		\item[SM4:] $\left(\Wen \approx 1.53\,\&\,\chi \approx 0.08\right)$ Experimental video of Case II for hexadecane drops: bouncing of the impacting drop followed by lift-off of the sessile drop.
		\item[SM5:] $\left(\Wen = 1.50\,\&\,\chi = 0.08\right)$ Simulation video of Case II for hexadecane drops: bouncing of the impacting drop followed by lift-off of the sessile drop.
		\item[SM6:] $\left(\Wen = 1.50\,\&\,\chi = 0.08\right)$ Simulation video showing velocity vectors of Case II for hexadecane drops: bouncing of the impacting drop followed by lift-off of the sessile drop. The two-dimensional contour represents the slice $Y = 0$. Time is normalized by the capillary time scale, $\tau_{\rho\gamma} = \sqrt{\left(\rho_l R_0^3\right)/\gamma}$.
		
		\item[SM7:] $\left(\Wen = 1.50\,\&\,\chi = 0.08\right)$ Experimental video of Case III for hexadecane drops: sliding-off of the impacting drop on top of the sessile drop followed by lift-off of the sessile drop.
		\item[SM8:] $\left(\Wen = 1.50\,\&\,\chi = 0.25\right)$ Simulation video of Case III for hexadecane drops: sliding-off of the impacting drop on top of the sessile drop followed by lift-off of the sessile drop.
		\item[SM9:] $\left(\Wen = 1.50\,\&\,\chi = 0.25\right)$  Simulation video showing velocity vectors of Case III for hexadecane drops: sliding-off of the impacting drop on top of the sessile drop followed by lift-off of the sessile drop. The two-dimensional contour represents the slice $Y = 0$. Time is normalized by the capillary time scale, $V_{\rho\gamma} = \sqrt{\left(\rho_l R_0^3\right)/\gamma}$.
		
		\item[SM10:] $\left(\Wen = 1.50\,\&\,\chi = 0.25\right)$ Experimental video of Case IV for hexadecane drops: sliding-off of the impacting drop on top of the sessile drop followed by its lift-off. In this case, the sessile drop stays on the substrate .
		\item[SM11:] $\left(\Wen \approx 1.50\,\&\,\chi \approx 0.625\right)$ Simulation video of Case IV for hexadecane drops: sliding-off of the impacting drop on top of the sessile drop followed by its lift-off. In this case, the sessile drop stays on the substrate.
		\item[SM12:] $\left(\Wen \approx 1.50\,\&\,\chi \approx 0.625\right)$ Simulation video showing velocity vectors of Case IV for hexadecane drops: sliding-off of the impacting drop on top of the sessile drop followed by its lift-off. In this case, the sessile drop stays on the substrate. The two-dimensional contour represents the slice $Y = 0$. Time is normalized by the capillary time scale, $\tau_{\rho\gamma} = \sqrt{\left(\rho_l R_0^3\right)/\gamma}$.
		
		\item[SM13:] $\left(\Wen \approx 5.84\,\&\,\chi \approx 0.08\right)$ Experimental video of Case V for hexadecane drops: coalescence of drops and lift-off of coalesced drop.
		\item[SM14:] $\left(\Wen \approx 1.43\,\&\,\chi \approx 0.03\right)$ Experimental video of Case VI for hexadecane drops: coalescence of drops and coalesced drop remains on the substrate.
	\end{enumerate}

%\newpage
%\thispagestyle{empty}

\begin{figure*}
	\centering
	\includegraphics[width=\textwidth]{Ch4Figures/QRcodesChapter4.eps}
\end{figure*}

%\newpage
%\thispagestyle{empty}

\end{subappendices}




